% Created 2023-06-14 Wed 10:54
% Intended LaTeX compiler: pdflatex
\documentclass[11pt]{book}
\usepackage[utf8]{inputenc}
\usepackage[T1]{fontenc}
% \usepackage{graphicx}
% \usepackage{grffile}
% \usepackage{longtable}
% \usepackage{wrapfig}
% \usepackage{rotating}
% \usepackage[normalem]{ulem}
% \usepackage{amsmath}
% \usepackage{textcomp}
% \usepackage{amssymb}
% \usepackage{capt-of}
% \usepackage{hyperref}

\author{Gabriel Bénédict}
\date{\today}
\title{AI for personalization: from predictive to generative modeling.}

\begin{document}

\maketitle
\tableofcontents
\chapter{Acknowledgments}
\chapter{Introduction}

Personalization on streaming platforms is oftentimes perceived as a purely predictive phenomenon: we propose to view it as a comprehensive and responsible generative approach, throughout a pipeline. We introduce RecFusion to issue recommendations in a generative way with diffusion models, as part of the nascent Generative Information Retrieval field. For these recommendations, we propose a method to generate personalized stills from movies, with sigmoidF1. We show that the resulting interactions on platforms are also dependent on implicit data hidden from a web analytics platform, with our intent-satisfaction analysis. At the end of the pipeline, we propose to ensure normative diversity in the issued recommendations with our RADio metrics framework.

\section{Scope and Research Questions}

\section{Main Contributions}

\section{Thesis Overview}

\section{Origins}





\end{document}
%%% Local Variables:
%%% mode: latex
%%% TeX-master: t
%%% End:
