% !TEX root = ../thesis-main.tex

\section{Research Outline and Questions}
\label{section:introduction:rqs}


We scope the thesis around four research questions, each corresponding to a chapter in the thesis.

A streaming platform ()

Personalization on streaming platforms is oftentimes perceived as a purely predictive phenomenon: we propose to view it as a comprehensive and responsible generative approach, throughout a pipeline. We introduce RecFusion to issue recommendations in a generative way with diffusion models, as part of the nascent Generative Information Retrieval field. For these recommendations, we propose a method to generate personalized stills from movies, with sigmoidF1. We show that the resulting interactions on platforms are also dependent on implicit data hidden from a web analytics platform, with our intent-satisfaction analysis. At the end of the pipeline, we propose to ensure normative diversity in the issued recommendations with our RADio metrics framework.


\begin{question}
  Can we use diffusion to do recommendation in the classical user-item matrix setting?
\end{question}

Is Bernoulli diffusion a suitable forward and backward process theoretically and empirically on binary data?

\begin{question}
  Is there a way we can generate personalized posters and stills for each item on a streaming platform?
\end{question}

Can we formulate loss function that accommodates for multilabel classification at training time and operates on the whole batch to balance confusion matrix entries?

\begin{question}
  Are users' intents together with their behavioral data useful signals to predict or explain satisfaction on a video streaming platform?
\end{question}

Can we use Bayesian posterior draws to meaningfully draw conclusions from data?

\begin{question}
  Can we formulate a divergence metric that measures the normative diversity of recommendations?
\end{question}

This work applies to news recommendation but trivially generalizes to any domain that has categories (e.g. video streaming with movie genres).

Can we formulate a divergence metric that is distributional and rank-aware?


%ref to sections with \acrodef{rq:focus}[\ref{rq:focus}]{}









%%% Local Variables:
%%% mode: latex
%%% TeX-master: "../thesis-main"
%%% End:
