% !TEX root = ../thesis-main.tex

\acrodef{rq:recfusion}[\ref{rq:recfusion}]{Can we use diffusion to do recommendation in the classical user-item matrix setting?}

\acrodef{rq:sigmoidf1}[\ref{rq:sigmoidf1}]{Is there a way we can generate personalized thumbnails for each item on a streaming platform?}

\acrodef{rq:intent}[\ref{rq:intent}]{Are users' intents together with their behavioral data useful signals to predict or explain satisfaction on a video streaming platform?}

\acrodef{rq:radio}[\ref{rq:radio}]{Can we formulate a divergence metric that measures the normative diversity of recommendations?}

\section{Research Outline and Questions}
\label{section:introduction:rqs}


We scope the thesis around four research questions, each corresponding to a research chapter in the thesis.

% https://sl.bing.net/c5jX2Jw4SHs

Personalization on streaming platforms is often seen as a way of predicting what users want to watch based on their preferences and behavior. Personalization can also be seen as a creative process (see recent generative recommendation models~\cite{llmRecNews, llmRec, genirRec}) with ethical impact (see recent literature on responsible recommendation~\cite{helberger, normalize, fairChatGPTReco}) that involves generating new content and experiences for users through a pipeline.

\noindent \mdr{What are the assumptions that you are making? What is the context?}
\gab{what kind of assumptions are expected here? I can think of a lot of assumptions regarding the model/data/platform/pipeline?}

Our first research question addresses the entry point of the user journey on a personalized platform, namely recommendations on the home page (the page where a user lands first on a streaming platform).
%
\begin{enumerate}[label=\textbf{RQ\arabic*},ref={RQ\arabic*},resume,leftmargin=*]
	\item \acl{rq:recfusion}\label{rq:recfusion}
\end{enumerate}
%



Traditionally, recommender systems directly retrieve content from the library to the user. Alternatively, user instructions and feedback are fed to a generator of personalized content, before retrieving and ranking from that new library of generated content, according to the recent generative recommendation paradigm~\cite{generativeReco}. According to its instigators, the paradigm covers individual content generated from scratch (like diffusion based image creation) or a recommendation of content that is created in a generative way (Large Language Models (LLMs) for recommendation). Somewhat combining the two concepts, we investigate how diffusion models can be used to generate a list of recommended content. Diffusion has been applied to images, music and other modalities. Unlike these, the classical recommendation setting of the user-item matrix~\cite{MF} does not entail spatial relationships between data points: contrary to pixels on an image, there is no information encoded in the allocation of users and item on a matrix. We illustrate this in RecFusion~\cite{recfusion}, where we first use Unets~\cite{unet} to fit data in a spatial way, before going back to the classical recommendation neural setting of feeding data user-by-user. For this one-dimensional user vector, we provide a proof and first experiments to show that a binomial (Bernoulli) diffusion process is viable.

After recommendation, the next step of our pipeline caters to the display of recommended videos via thumbnails.
%
%Can diffusion still be applied in this setting? The binary nature of the classical user-item matrix formulation inspired a second more theoretical research question: \emph{can Bernoulli diffusion be a suitable forward and backward process theoretically and empirically on binary data?}.
%
\begin{enumerate}[label=\textbf{RQ\arabic*},ref={RQ\arabic*},resume,leftmargin=*]
	\item \acl{rq:sigmoidf1}\label{rq:sigmoidf1}
\end{enumerate}
%
Personalization can be seen at different levels of granularity: from targeting user segments (into interests, age groups, etc.) to targeting single users differently. For this research question, we are interested in how thumbnails (i.e. static images) can be classified into different categories, more than knowing if we can target each single user. We therefore opt for a least granular option: we assume that each user has a favorite genre. We can provide a thumbnail personalized to that genre (e.g., show a romantic scene from an action movie, if the favorite genre is romance). Given editorial or automatically selected candidates for thumbnails, we wish to display the one that is most closely associated with the user's preferences. This reduces to a multilabel classification problem: given an image, predict one, or many, genre(s). When thinking about classification, the confusion matrix~\cite{confusionMatrix} – with its false positive, true positive, false negative and true positive quadrants – is a classic way to build evaluation metrics. But these metrics are hardly used at training time. We think it is because these quadrant values require counting, which is not differentiable at training time for gradient descent~\cite{sgd1, sgd2}. We propose a way to build surrogates to these count metrics via sigmoid functions. More precisely, we look at maximizing for the F1 score via our sigmoidF1 surrogate loss function~\cite{sigmoidf1}, as a multilabel classification loss over an entire batch. We show that this improves on classical image and text benchmarks with classical backbones (CNN~\cite{cnn} and transformers~\cite{attention}).

Recommendations and thumbnails are what primes the users interactions with the platform. This relates to the next step in our pipeline.
%
%\emph{Can we formulate a loss function that accommodates for multilabel classification at training time and operates on the whole batch to balance confusion matrix entries?
%
\begin{enumerate}[label=\textbf{RQ\arabic*},ref={RQ\arabic*},resume,leftmargin=*]
	\item \acl{rq:intent}\label{rq:intent}
  \end{enumerate}
%
Streaming platforms have access to user implicit (clicks, scrolls, time on the platform, etc.) and explicit (ratings and bookmarks) feedback via their personalization pipeline. Some of the user behaviors will remain forever hidden from the platform though for privacy or technical reasons (e.g., how many people sit in front of the device, the content the user consumes on other platforms). Among them, we explore user intents of a video streaming platform. Previous work has defined intents for music~\cite{spotifyIntent}; we propose to define them for video and propose a transparent approach by revealing our survey design, code and simulated data. In~\cite{spotifyIntent}, logistic regression was used to predict satisfaction based on intents and behavioral data. We propose to use random forests and Bayesian hierarchical modeling to enhance accuracy and interpretability respectively.

Finally, we close off our pipeline with a responsible approach to diversity.
%
%Can we use Bayesian posterior draws to meaningfully draw conclusions from data?
%
\begin{enumerate}[label=\textbf{RQ\arabic*},ref={RQ\arabic*},resume,leftmargin=*]
	\item \acl{rq:radio}\label{rq:radio}
\end{enumerate}
%
Videos and especially news platforms serve content that is opinionated. Over time, platforms have been growing their engineering teams to cover more and more of the user journey stages (home page, title fonts, watch/read next etc.)~\cite{NetflixReco}, with more and more powerful and sometimes generative models. The user is thus influenced by the platform's algorithm and thus the platform's explicit or implicit norms and values. Can we empower a video/news platform to measure its ability to cater to its norms and values? We would like to account for how a platform means to properly inform citizens (as defined by~\cite{helberger}) and any form of diversity metric (topic, presence of alternative voices, complexity of the text, etc.). RADio~\cite{radio}, the framework we propose caters to these normative aspects but also to the specific recommendation context: RADio is rank aware and caters for any kind of discrete distribution via a our proposed rank-aware Jensen-Shannon divergence~\cite{js}. This chapter is focused on news recommendation but trivially generalizes to any domain that has categories (e.g., video streaming with movie genres).

%https://sl.bing.net/ckJiDKkaL6a
Our research questions have been outlined in this section. The main contributions of this thesis will be summarized in the next section.

%ref to sections with \acrodef{rq:focus}[\ref{rq:focus}]{}









%%% Local Variables:
%%% mode: latex
%%% TeX-master: "../thesis-main"
%%% End:
