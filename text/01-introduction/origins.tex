% !TEX root = ../thesis-main.tex

\section{Origins}
\label{section:introduction:origins}

We list the publications that are the origins of each chapter below.

\begin{enumerate}[label=\textbf{Chapter~\arabic*},align=left]
\setcounter{enumi}{0}

\item is based on the following paper:
\begin{itemize}
\item \bibentry{recfusion}.
\end{itemize}
GB wrote the first draft, code, experiments and mathematical formulations. GB was helped all along the way via discussion with the coauthors. Coauthors then edited the first draft together with GB. GB did most of the writing.


\item is based on the following paper:
\begin{itemize}
\item \bibentry{sigmoidf1}.
\end{itemize}
GB wrote the first draft, code, experiments and mathematical formulations. GB was helped all along the way via discussion with the coauthors. Coauthors then edited the first draft together with GB. GB did most of the writing.


\item is based on the following paper:
\begin{itemize}
\item \bibentry{intent}.
\end{itemize}
GB wrote the first draft, code, experiments and mathematical formulations. GB was helped all along the way via discussion with the coauthors. Coauthors then edited the first draft together with GB. GB did most of the writing.


\item is based on the following paper:
\begin{itemize}
\item \bibentry{radio}.
\end{itemize}
GB, together with SV, wrote the first draft, code, experiments and mathematical formulations. GB was helped all along the way via discussion with the coauthors. Coauthors then edited the first draft together with GB. GB and SV did most of the writing.


\end{enumerate}

%\todo{\noindent%
%We note that parts of Chapter~\ref{chapter:introduction} are based on the following paper:
%\begin{itemize}
%\end{itemize}}

\noindent%
The writing of this thesis also benefited from work on the following publications:

\begin{itemize}
\item \bibentry{genir}.
\item \bibentry{trec}.
\item \bibentry{gans}. 
\end{itemize}
%%% Local Variables:
%%% mode: latex
%%% TeX-master: "../thesis-main"
%%% End:
