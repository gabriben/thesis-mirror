% !TEX root = ../thesis-main.tex

\chapter{Introduction}
\label{chapter:introduction}

% aided by this chatGPT conversation: https://chat.openai.com/share/2cbbe97e-207c-4cfb-8558-c576a155814a
% and this BING chat conversation: https://sl.bing.net/M0SlEtfmSG, https://sl.bing.net/f7cUBW51tam

check these https://dl.acm.org/doi/abs/10.1145/3460231.3474612\#sec-ref \tocite{}

Video streaming platforms have changed the way people consume and interact with digital media \tocite{}. Using machine learning to deliver personalized and recommended content to users based on their behavior and preferences is not something new~\cite{oldPersonalizationBehavior, oldPersonalizationSearch}. At first, personalization was restricted to email newsletters, then appeared on the platform in a form of 1 dimensional lists \tocite{}. Now personalization is in the ordering of the strip, the thumbnail, in the font title of the thumbnail, in search \tocite{}. The outcome is an entire \emph{user journey} (the user's perspective) steered by the platforms algorithms. For the purpose of this thesis, we will call this combination of algorithms, the \emph{personalization pipeline} (the platform's perspective). This pipeline is geared towards simple KPIs like number of minutes seen \tocite{} and churn rate \tocite{}, but is linked with a responsibility to balance longer term user satisfaction, content diversity, and ethical considerations \tocite{}.

\paragraph{The user journey} consists of several steps. First, users come to the platform with some \emph{intents} (e.g., binge-watching a series, finding a family-friendly movie, discovering new genres, etc.)~\cite{intent}. Then, they see a customized home page with various horizontal \emph{recommendation} strips (see Figure~\ref{fig:VLStrip}). Each strip contains videos with (sometimes personalized) \emph{thumbnails}. Over time, users interact with the platform and leave \emph{behavioral} signals (e.g., clicks, watch time, bookmarks, ratings, etc.). From the platform's perspective, deciphering how these behaviors, prompted by user intents, translate into overall \emph{satisfaction} remains a complex challenge. 

\begin{figure}[h]
  \centering
  \includegraphics[width=\textwidth]{01-introduction/images/VLHome_cropped.png}
  \caption{Videoland's Recommendation strips}
  \label{fig:VLStrip}
\end{figure}

\paragraph{The personalization pipeline} is our description of the accumulation of a platform's algorithms that cater for a better user journey. The pipeline includes collecting user data, analyzing user behavior. These signals, also called user feedback are often provided by an external analytics business and the signal granularity is limited by the amount of data and how much a streaming platform is willing to pay \tocite{}. It can range from number of item watched (just one data point per user session) all the way to recordings of all mouse movements (thousands of data points per session). With that session-level data, streaming platforms attempt to predict what the user will do, to adapt the user journey to the user: the next movie to watch, the subsequent logins, the midterm satisfaction, all the way to churn; in increasing order of prediction horizon. These predictive models are tested first offline and then evaluated online over several iterations and over time. In the evaluation phase, preferences, and interaction patterns are captured again as a feedback loop. Aside from the measurable user feedback signals, a platform can also take hidden signals into consideration. In this thesis, we give some attention to user intent (watching a next episode on a favorite show, looking for new content, bookmarking items for later viewership). Besides satisfying the users, the platform also has to ensure that the content it offers is \emph{diverse} and promotes representative. For example, promoting screenwriters of different genres, movies in different languages, a variety of movie genres. These concerns are revisited below.


%regarding satisfaction and diversity are related to the platform's perspective.


%building predictive models, testing different strategies, and evaluating the outcomes. These steps help to capture user intent, preferences, and interaction patterns; the input for personalization tools. This thesis presents some novel methods and tools for improving the pipeline along this user journey. 

In this thesis, we propose individual tools that map to the user journey above, for the steps of \emph{recommendations}, \emph{thumbnail} selection, \emph{intent-satisfaction} linking and \emph{diversity} measurement. Together, these tools form our proposed \textbf{personalization pipeline}. For \emph{recommendations}, we found a good amount of literature on diffusion models \cite{} for continuous 2D structured data but little-to-no literature on any relaxation of the above. We propose to explore formulation of diffusion models for 1D binary unstructured data. As for personalized \emph{thumbnails}, existing research in multilabel image classification is highly reliant on variations of the binary cross-entropy loss \cite{}, but we think that the multilabel setting (as opposed to the binary or multiclass setting) requires its own solution. For the next step, we identified a lack of a systematic approach for \emph{intent-satisfaction} studies, that would provide survey design, code and modern Bayesian approaches to the problem. Finally, we could not find a \emph{diversity} metric that is distribution agnostic and rank-aware, to adapt to any normative standpoints of a platform issuing news / movies recommendations.

%We identify a knowledge gap in the literature 
%\paragraph{knowledge gap...} list for each paper.

%The tools we propose here are: a generative model for creating diverse and relevant recommendation strips for the home page; a multilabel classifier for selecting optimal thumbnails for each video; a survey-based method for predicting user satisfaction from intent and behavior; and a set of metrics for measuring content diversity and avoiding unwanted biases.



In short, we focus on the video streaming platform ecosystem, exploring the challenges and opportunities of personalization, recommendation, and user behavior analysis. By combining survey methods, content modeling, adaptive testing, and behavioral analysis, this study aims to contribute to the development of video streaming platforms that can provide user satisfaction in a responsible way.


%% The research questions and sub questions
% !TEX root = ../thesis-main.tex

\acrodef{rq:recfusion}[\ref{rq:recfusion}]{Can we use diffusion to do recommendation in the classical user-item matrix setting?}

\acrodef{rq:sigmoidf1}[\ref{rq:sigmoidf1}]{Is there a way we can generate personalized thumbnails for each item on a streaming platform?}

\acrodef{rq:intent}[\ref{rq:intent}]{Are users' intents together with their behavioral data useful signals to predict or explain satisfaction on a video streaming platform?}

\acrodef{rq:radio}[\ref{rq:radio}]{Can we formulate a divergence metric that measures the normative diversity of recommendations?}

\section{Research Outline and Questions}
\label{section:introduction:rqs}


We scope the thesis around four research questions, each corresponding to a research chapter in the thesis.

% https://sl.bing.net/c5jX2Jw4SHs

Personalization on streaming platforms is often seen as a way of predicting what users want to watch based on their preferences and behavior. Personalization can also be seen as a creative process (see recent generative recommendation models~\cite{llmRecNews, llmRec, genirRec}) with ethical impact (see recent literature on responsible recommendation~\cite{helberger, normalize, fairChatGPTReco}) that involves generating new content and experiences for users through a pipeline.

\noindent \mdr{What are the assumptions that you are making? What is the context?}
\gab{what kind of assumptions are expected here? I can think of a lot of assumptions regarding the model/data/platform/pipeline?}

Our first research question addresses the entry point of the user journey on a personalized platform, namely recommendations on the home page (the page where a user lands first on a streaming platform).
%
\begin{enumerate}[label=\textbf{RQ\arabic*},ref={RQ\arabic*},resume,leftmargin=*]
	\item \acl{rq:recfusion}\label{rq:recfusion}
\end{enumerate}
%



Traditionally, recommender systems directly retrieve content from the library to the user. Alternatively, user instructions and feedback are fed to a generator of personalized content, before retrieving and ranking from that new library of generated content, according to the recent generative recommendation paradigm~\cite{generativeReco}. According to its instigators, the paradigm covers individual content generated from scratch (like diffusion based image creation) or a recommendation of content that is created in a generative way (Large Language Models (LLMs) for recommendation). Somewhat combining the two concepts, we investigate how diffusion models can be used to generate a list of recommended content. Diffusion has been applied to images, music and other modalities. Unlike these, the classical recommendation setting of the user-item matrix~\cite{MF} does not entail spatial relationships between data points: contrary to pixels on an image, there is no information encoded in the allocation of users and item on a matrix. We illustrate this in RecFusion~\cite{recfusion}, where we first use Unets~\cite{unet} to fit data in a spatial way, before going back to the classical recommendation neural setting of feeding data user-by-user. For this one-dimensional user vector, we provide a proof and first experiments to show that a binomial (Bernoulli) diffusion process is viable.

After recommendation, the next step of our pipeline caters to the display of recommended videos via thumbnails.
%
%Can diffusion still be applied in this setting? The binary nature of the classical user-item matrix formulation inspired a second more theoretical research question: \emph{can Bernoulli diffusion be a suitable forward and backward process theoretically and empirically on binary data?}.
%
\begin{enumerate}[label=\textbf{RQ\arabic*},ref={RQ\arabic*},resume,leftmargin=*]
	\item \acl{rq:sigmoidf1}\label{rq:sigmoidf1}
\end{enumerate}
%
Personalization can be seen at different levels of granularity: from targeting user segments (into interests, age groups, etc.) to targeting single users differently. For this research question, we are interested in how thumbnails (i.e. static images) can be classified into different categories, more than knowing if we can target each single user. We therefore opt for a least granular option: we assume that each user has a favorite genre. We can provide a thumbnail personalized to that genre (e.g., show a romantic scene from an action movie, if the favorite genre is romance). Given editorial or automatically selected candidates for thumbnails, we wish to display the one that is most closely associated with the user's preferences. This reduces to a multilabel classification problem: given an image, predict one, or many, genre(s). When thinking about classification, the confusion matrix~\cite{confusionMatrix} – with its false positive, true positive, false negative and true positive quadrants – is a classic way to build evaluation metrics. But these metrics are hardly used at training time. We think it is because these quadrant values require counting, which is not differentiable at training time for gradient descent~\cite{sgd1, sgd2}. We propose a way to build surrogates to these count metrics via sigmoid functions. More precisely, we look at maximizing for the F1 score via our sigmoidF1 surrogate loss function~\cite{sigmoidf1}, as a multilabel classification loss over an entire batch. We show that this improves on classical image and text benchmarks with classical backbones (CNN~\cite{cnn} and transformers~\cite{attention}).

Recommendations and thumbnails are what primes the users interactions with the platform. This relates to the next step in our pipeline.
%
%\emph{Can we formulate a loss function that accommodates for multilabel classification at training time and operates on the whole batch to balance confusion matrix entries?
%
\begin{enumerate}[label=\textbf{RQ\arabic*},ref={RQ\arabic*},resume,leftmargin=*]
	\item \acl{rq:intent}\label{rq:intent}
  \end{enumerate}
%
Streaming platforms have access to user implicit (clicks, scrolls, time on the platform, etc.) and explicit (ratings and bookmarks) feedback via their personalization pipeline. Some of the user behaviors will remain forever hidden from the platform though for privacy or technical reasons (e.g., how many people sit in front of the device, the content the user consumes on other platforms). Among them, we explore user intents of a video streaming platform. Previous work has defined intents for music~\cite{spotifyIntent}; we propose to define them for video and propose a transparent approach by revealing our survey design, code and simulated data. In~\cite{spotifyIntent}, logistic regression was used to predict satisfaction based on intents and behavioral data. We propose to use random forests and Bayesian hierarchical modeling to enhance accuracy and interpretability respectively.

Finally, we close off our pipeline with a responsible approach to diversity.
%
%Can we use Bayesian posterior draws to meaningfully draw conclusions from data?
%
\begin{enumerate}[label=\textbf{RQ\arabic*},ref={RQ\arabic*},resume,leftmargin=*]
	\item \acl{rq:radio}\label{rq:radio}
\end{enumerate}
%
Videos and especially news platforms serve content that is opinionated. Over time, platforms have been growing their engineering teams to cover more and more of the user journey stages (home page, title fonts, watch/read next etc.)~\cite{NetflixReco}, with more and more powerful and sometimes generative models. The user is thus influenced by the platform's algorithm and thus the platform's explicit or implicit norms and values. Can we empower a video/news platform to measure its ability to cater to its norms and values? We would like to account for how a platform means to properly inform citizens (as defined by~\cite{helberger}) and any form of diversity metric (topic, presence of alternative voices, complexity of the text, etc.). RADio~\cite{radio}, the framework we propose caters to these normative aspects but also to the specific recommendation context: RADio is rank aware and caters for any kind of discrete distribution via a our proposed rank-aware Jensen-Shannon divergence~\cite{js}. This chapter is focused on news recommendation but trivially generalizes to any domain that has categories (e.g., video streaming with movie genres).

%https://sl.bing.net/ckJiDKkaL6a
Our research questions have been outlined in this section. The main contributions of this thesis will be summarized in the next section.

%ref to sections with \acrodef{rq:focus}[\ref{rq:focus}]{}









%%% Local Variables:
%%% mode: latex
%%% TeX-master: "../thesis-main"
%%% End:


%% Lists the main contributions of the thesis
% !TEX root = ../thesis-main.tex

\section{Main Contributions}
\label{section:introduction:contributions}

In this section, we summarize the main contributions of this thesis. We separate theoretical from artifact (tool and experimental design) contributions.

\gab{I found this 'artifact' expression \href{https://courses.cs.washington.edu/courses/cse510/16wi/readings/wobbrock\_7contributions\_submitted.pdf}{here}, but I am not sure it is commonly used}

\subsection*{Theoretical Contributions}

\begin{itemize}
\item Diffusion applied on unstructured data, where there is no spatial dependency (Chapter~\ref{chapter:research-recfusion}).
\item Diffusion for binary and/or 1D data: A demonstration that KL divergence is also suited for binary data and that the Bernouilli Markov Process has the same properties as its Gaussian counterpart (Chapter~\ref{chapter:research-recfusion}).
\item A multilabel loss function that accounts for all examples in a batch (Chapter~\ref{chapter:research-sigmoidf1}).
\item An F1 score surrogate as a loss function (Chapter~\ref{chapter:research-sigmoidf1}).
\item An account of the current limitations and underreporting of thresholding at inference time (Chapter~\ref{chapter:research-sigmoidf1}).
\item a proposal of typical intents for a video streaming that we divide into explorative and decisive categories (Chapter~\ref{chapter:research-intent}).
\item A diversity metric that is versatile to any normative concept and expressed as the divergence between two (discrete) distributions, rank-aware and mathematically grounded in distributional divergence statistics (Chapter~\ref{chapter:research-radio}).
\end{itemize}

\subsection*{Artifact Contributions}

\begin{itemize}
\item Frequentist logistic regression model, we test Bayesian multilevel models for visualization and explanations, along with random forests for improved accuracy (Chapter~\ref{chapter:research-intent}).
\item A reproducibility study from music to video streaming platforms of intent-satisfaction modeling (this time with code and synthetic data) (Chapter~\ref{chapter:research-intent}).
\item In-app survey design for a medium size streaming platform ($\sim$1 million users) and corresponding synthetic data (Chapter~\ref{chapter:research-intent}).
\item A metadata enrichment pipeline (e.g., sentiment analysis, named entity recognition) to extract normative concepts from news articles (Chapter~\ref{chapter:research-radio}).
\end{itemize}
%%% Local Variables:
%%% mode: latex
%%% TeX-master: "../thesis-main"
%%% End:


%% Overview of the thesis; what is described in which chapter
% !TEX root = ../thesis-main.tex

\section{Thesis Overview}
\label{section:introduction:overview}


%https://sl.bing.net/cjw62MnC0E8

This first chapter introduces the main topics and goals of this thesis, and suggests some possible ways to read it. The thesis has six chapters in total, and this is the first one. The following four chapters each address one of the research questions that we presented in Section~\ref{section:introduction:rqs}. Each chapter is based on a paper (see Section~\ref{section:introduction:origins} below), and can be read on its own. If the reader is interested in the entire thesis, we recommend following the original order of chapters, as they follow the \emph{user journey} and its respective \emph{personalization pipeline} on a streaming platform. The final chapter summarizes the main findings and contributions of this thesis, and proposes some future research directions.

%%% Local Variables:
%%% mode: latex
%%% TeX-master: "../thesis-main"
%%% End:


%% Describes the papers from which the chapters originate
% !TEX root = ../thesis-main.tex

\section{Origins}
\label{section:introduction:origins}

We list the publications that are the origins of each chapter below.

\begin{enumerate}[label=\textbf{Chapter~\arabic*},align=left]
\setcounter{enumi}{1}

\item is based on the following paper:
\begin{itemize}
\item \bibentry{recfusion}.
\end{itemize}
GB wrote the first draft, code, mathematical formulations, designed and ran experiments. GB was helped all along the way via discussion with the coauthors. Coauthors then edited the first draft together with GB. GB did most of the writing.

\item is based on the following paper:
\begin{itemize}
\item \bibentry{sigmoidf1}.
\end{itemize}
GB wrote the first draft, code, mathematical formulations, designed and ran experiments. GB was helped all along the way via discussion with the coauthors. Coauthors then edited the first draft together with GB. GB did most of the writing.

\item is based on the following paper:
\begin{itemize}
\item \bibentry{intent}.
\end{itemize}
GB wrote the first draft, code, mathematical formulations, designed and ran experiments. GB was helped all along the way via discussion with the coauthors. Coauthors then edited the first draft together with GB. GB did most of the writing.

\item is based on the following paper:
\begin{itemize}
\item \bibentry{radio}.
\end{itemize}
GB, together with SV, wrote the first draft, code, mathematical formulations, designed and ran experiments. GB was helped all along the way via discussion with the coauthors. Coauthors then edited the first draft together with GB. GB and SV did most of the writing.


\end{enumerate}

%\todo{\noindent%
%We note that parts of Chapter~\ref{chapter:introduction} are based on the following paper:
%\begin{itemize}
%\end{itemize}}

\noindent%
The writing of this thesis also benefited from work on the following publications:

\begin{itemize}
\item \bibentry{genir}.
\item \bibentry{trec}.
\item \bibentry{gans}.
\end{itemize}
%%% Local Variables:
%%% mode: latex
%%% TeX-master: "../thesis-main"
%%% End:

%%% Local Variables:
%%% mode: latex
%%% TeX-master: "../thesis-main"
%%% End:
