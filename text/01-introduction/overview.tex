% !TEX root = ../thesis-main.tex

\section{Thesis Overview}
\label{section:introduction:overview}


%https://sl.bing.net/cjw62MnC0E8

This chapter introduces the main topics and goals of this thesis, and suggests some possible ways to read it. The thesis has seven chapters in total, and this is the first one. The following five chapters each address one of the research questions that we presented in Section~\ref{section:introduction:rqs}. Each chapter is based on a paper (see Section~\ref{section:introduction:origins} below), and can be read on its own. If the reader is interested in the entire thesis, we recommend following the original order of chapters, as they follow the user journey on a streaming platform. The final chapter summarizes the main findings and contributions of this thesis, and proposes some future research directions.


\if
This thesis is organized into three parts, each part can be read independently. 

The first part focuses on proposing new algorithms for explaining predictions from ML models. 
Specifically, we propose methods for generating counterfactual explanations for tree-based models (Chapter~\ref{chapter:research-focus}), and for graph-based models (Chapter~\ref{chapter:research-cfgnn}). 
These methods can be applied on any tree- or graph-based model, respectively.

The second part focuses on the interaction between ML explanations and the users who consume them. 
We propose a method for explaining errors in forecasting predictions (Chapter~\ref{chapter:research-mcbrp}). 
To evaluate our method, we propose a user study with both objective and subjective components, where we contrast and compare the results between two types of users: researchers and practitioners. 


In the third part of the thesis, we shift our focus from translating knowledge about individual predictions to transferring knowledge to the next generation of researchers. 
We propose a course setup for teaching about responsible AI topics to a graduate-level audience and reflect on our learnings from past implementations of the course at the University of Amsterdam (Chapter~\ref{chapter:research-pedagogy}). 
\fi


%%% Local Variables:
%%% mode: latex
%%% TeX-master: "../thesis-main"
%%% End:
