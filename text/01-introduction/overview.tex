% !TEX root = ../thesis-main.tex

\section{Thesis Overview}
\label{section:introduction:overview}

This thesis is organized into three parts, each part can be read independently. 

The first part focuses on proposing new algorithms for explaining predictions from ML models. 
Specifically, we propose methods for generating counterfactual explanations for tree-based models (Chapter~\ref{chapter:research-focus}), and for graph-based models (Chapter~\ref{chapter:research-cfgnn}). 
These methods can be applied on any tree- or graph-based model, respectively.

The second part focuses on the interaction between ML explanations and the users who consume them. 
We propose a method for explaining errors in forecasting predictions (Chapter~\ref{chapter:research-mcbrp}). 
To evaluate our method, we propose a user study with both objective and subjective components, where we contrast and compare the results between two types of users: researchers and practitioners. 


In the third part of the thesis, we shift our focus from translating knowledge about individual predictions to transferring knowledge to the next generation of researchers. 
We propose a course setup for teaching about responsible AI topics to a graduate-level audience and reflect on our learnings from past implementations of the course at the University of Amsterdam (Chapter~\ref{chapter:research-pedagogy}). 


