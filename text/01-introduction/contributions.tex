% !TEX root = ../thesis-main.tex

\section{Main Contributions}
\label{section:introduction:contributions}

In this section, we summarize the main contributions of this thesis. We separate theoretical from artifact (tool and experimental design) contributions.

\subsection*{Theoretical Contributions}

\begin{itemize}
\item Diffusion applied on unstructured data, where there is no spatial dependency (Chapter~\ref{chapter:research-recfusion}).
\item Diffusion for binary and/or 1D data: A demonstration that KL divergence is also suited for binary data and that the Bernouilli Markov Process has the same properties as its Gaussian counterpart (Chapter~\ref{chapter:research-recfusion}).
\item A multilabel loss function that accounts for all examples in a batch (Chapter~\ref{chapter:research-sigmoidf1}).
\item An F1 score surrogate as a loss function (Chapter~\ref{chapter:research-sigmoidf1}).
\item An account of the current limitations and underreporting of thresholding at inference time (Chapter~\ref{chapter:research-sigmoidf1}).
\item a proposal of typical intents for a video streaming that we divide into explorative and decisive categories (Chapter~\ref{chapter:research-intent}).
\item A diversity metric that is versatile to any normative concept and expressed as the divergence between two (discrete) distributions, rank-aware and mathematically grounded in distributional divergence statistics (Chapter~\ref{chapter:research-radio}).
\end{itemize}

\subsection*{Artifact Contributions}

\begin{itemize}
\item Frequentist logistic regression model, we test Bayesian multilevel models for visualization and explanations, along with random forests for improved accuracy (Chapter~\ref{chapter:research-intent}).
\item A reproducibility study from music to video streaming platforms of intent-satisfaction modeling (this time with code and synthetic data) (Chapter~\ref{chapter:research-intent}).
\item In-app survey design for a medium size streaming platform ($\sim$1 million users) and corresponding synthetic data (Chapter~\ref{chapter:research-intent}).
\item A metadata enrichment pipeline (e.g., sentiment analysis, named entity recognition) to extract normative concepts from news articles (Chapter~\ref{chapter:research-radio}).
\end{itemize}
%%% Local Variables:
%%% mode: latex
%%% TeX-master: "../thesis-main"
%%% End:
