% !TEX root = ../thesis-main.tex
\chapter{Normative Diversity}
\label{chapter:research-radio}

\begin{figure}[ht]
  \centering
  \includegraphics{images/pipeline_step4.pdf}
  \caption{The final step of the personalization pipeline}
  \label{fig:pip4}
\end{figure}

\footnote[]{This chapter was published at the ACM Conference on Recommender Systems (RecSys 2022) under the title ``RADio – Rank-Aware Divergence Metrics to Measure Normative Diversity in News Recommendations'' \citep{radio}, , where it won a best paper runner up award.}
\acresetall

In this chapter, we address the following research question:

\medskip
\noindent
\textbf{\ref{rq:radio}:} \acl{rq:radio}
\medskip

\noindent



\section{Introduction}
\subimport{RADio-TORS/sections}{01_intro}
\label{sec:introduction}

\subimport{RADio-TORS/sections}{02_related_work}

\section{Operationalizing Normative Diversity for News Recommendation}
\label{sec:method}
\subimport{RADio-TORS/sections}{03_method}

\label{sec:metrics_implementation}
\subimport{RADio-TORS/sections}{04_metrics}

\section{Experimental Setup} 
\label{sec:experiments}
\subimport{RADio-TORS/sections}{05_experiments}

\section{Experimental Results} 
\label{sec:results}
\subimport{RADio-TORS/sections}{06_results}

\section{The Effects of Metric Design Choices}
\subimport{RADio-TORS/sections}{06_results_calibration}

\section{Discussion}
\label{sec:discussion}
\subimport{RADio-TORS/sections}{07_discussion}

\section{Conclusion}
\label{sec:conclusions}
\subimport{RADio-TORS/sections}{08_conclusions}



\section*{Reproducibility}
To facilitate the reproducibility of the work in this chapter, our code is available at \url{https://github.com/svrijenhoek/RADio}.

%%% Local Variables:
%%% mode: latex
%%% TeX-master: "../thesis-main"
%%% End:
