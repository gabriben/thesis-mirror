%!TEX root = ../main.tex


\section{Conclusion}
\label{section:fact-conclusion}

In this chapter, we share our setup for the FACT-AI course at \OurUniversity{}, which teaches FACT-AI topics through reproducibility. 
The course set out to give students 
\begin{enumerate*}[label=(\roman*)]
    \item an understanding of FACT-AI topics, 
    \item an understanding of algorithmic harm, 
    \item familiarity with recent FACT-AI methods, and
    \item an opportunity to reproduce FACT-AI solutions, 
\end{enumerate*}
through a combination of lectures, paper discussion sessions and a reproducibility project. 
%
Through their projects, our students engaged with the open-source community by creating a public code repository (in the 2019--2020 iteration), as well as with the research community via successful submissions to the ML Reproducibility Challenge (in the 2020--2021 iteration). 
We also detail how the 2020--2021 iteration brought about its own unique set of challenges due to the COVID-19 pandemic. 

In this course, we illustrate that reproducibility should be viewed as a fundamental component of FACT-AI. 
We received very positive feedback on teaching FACT-AI topics through reproducibility. We believe this was an excellent fit for our students, which not only helped motivate them for the duration of the course, but also helped them develop skills that will be essential in their future research careers, whether in the private or public sector. 


With this final chapter, we answer \textbf{\ref{rq:pedagogy}}: we can use reproducibility as a mechanism for teaching responsible AI concepts to a technical, research-oriented audience. 
Structuring the course around a reproducibility project gives students the opportunity to learn about responsible AI concepts, such as explainability, in a hands-on manner. 
Since the publication of the paper on which this chapter is based \citep{lucic2022reproducibility}, we ran another iteration of the FACT-AI course in 2021--2022 under the same setup as the previous year, where students submitted their reports to the 2022 edition of the ML Reproducibility Challenge. 
21 of the 43 papers accepted to the ML Reproducibility Challenge in 2022 were from students in the FACT-AI course. 
These also included some awards: the Best Paper Award was awarded to a group from the FACT-AI course as well as 2 of the 4 Outstanding Paper Awards. 
We believe this indicates that our course setup can serve as a starting point for effective participation in the broader ML research community. 