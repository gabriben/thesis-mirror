%!TEX root = ../main.tex


\section{Introduction}
\label{section:fact-introduction}

For several decades, \OurUniversity{} has offered a research-oriented Master of Science (MSc) program in AI. 
The main focus of the program is on the technical machine learning (ML) aspects of the major sub-fields of AI, such as computer vision, information retrieval, natural language processing, and reinforcement learning.
One of the most recent additions to the MSc AI curriculum is a mandatory course called \emph{\ac{FACT-AI}}. 
This course was first taught during the 2019--2020 academic year and focuses on teaching FACT-AI topics through the lens of reproducibility. 
The main project involves students working in groups to re-implement existing FACT-AI algorithms from papers in top AI venues. 
There are approximately 150 students enrolled in the course each year. 

The motivation for the course came from the MSc AI students themselves, who often play an important role in shaping the curriculum in order to meet the evolving requirements of researchers in both academia and industry. 
As the influence of AI on decision making is becoming increasingly prevalent in day-to-day life, there is a growing consensus that stakeholders who take part in the design or implementation of AI algorithms should reflect on the ethical ramifications of their work, including developers and researchers \citep{campbell2021_responsible}. 
This is especially true in situations where data-driven AI systems affect some demographic sub-groups differently than others \citep{propublica, o2017ivory}.
As a result, our students have shown an increased interest in the ethical issues surrounding AI systems and requested that the university put together a new course focusing on responsible AI.

Since our MSc AI program is characterized by a strong emphasis on understanding, developing, and building AI algorithms, we believe that a new course on responsible AI in this program should also have a hands-on approach.
The course is designed to address technical aspects of key areas in responsible AI:
\begin{enumerate*}[label=(\roman*)]
\item fairness, 
\item accountability, 
\item confidentiality, and 
\item transparency,  
\end{enumerate*}
which we operationalize through a reproducibility project. 
We believe a strong emphasis on reproducibility is important from both an educational point of view and from the point of view of the AI community, since the (lack of) reproducible results has become a major point of critique in AI \citep{hutson2018artificial}. 
Moreover, the starting point of almost any junior AI researcher (and most AI research projects in general) is re-implementing existing methods as baselines. 
The FACT-AI course is situated at a point in the program where students have learned the basics of ML and are ready to start experimenting with, and building on top of, state-of-the-art algorithms. 
Given that our MSc AI program is fairly research-oriented, it is important for students to experience the process of reproducing work done by others (and how difficult this is) at an early stage in their careers. 
We also believe reproducibility is a fundamental component of FACT-AI: the cornerstone of fair, accountable, confidential and transparent AI systems is having correct and reproducible results. 
Without reproducibility, it is unclear how to judge if a decision-making algorithm adheres to any of the FACT principles. 

In the 2019--2020 academic year, we operationalized our learning ambitions regarding reproducibility by publishing a public repository with selected code implementations and corresponding reports from the group projects.
In the 2020--2021 academic year, we took the projects one step further and encouraged students to submit to the ML Reproducibility Challenge,\footnote{\url{https://paperswithcode.com/rc2020}} a competition that solicits reproducibility reports for papers published in conferences such as NeurIPS, ICML, ICLR, ACL, EMNLP, CVPR and ECCV. 
Although the challenge broadly focuses on all papers submitted to these conferences, we focus exclusively on papers covering FACT-AI topics in our course. 
Submitting to the challenge gives students a chance to experience the whole AI research pipeline, from running experiments, to writing rebuttals, to receiving the official notifications. 
Of the 23 papers that were accepted to the ML Reproducibility Challenge in 2021, 9 came from groups in the FACT-AI course. 

In this chapter, we describe the FACT-AI course at \OurUniversity{}: a one month, full-time course based on examining ethical issues in AI using reproducibility as a pedagogical tool. 
Students work in groups to re-implement (and possibly extend) existing algorithms from top AI venues on \ac{FACT-AI} topics. 
The course also includes lectures that cover the high-level principles of FACT-AI topics, as well as paper discussion sessions where students read and digest prominent FACT-AI papers. 
In this chapter, we outline the setup for the \ac{FACT-AI} course and the experiences we had while running the course during the 2019--2020 and 2020--2021 academic years at \OurUniversity{}. 

The remainder of this chapter is structured as follows. 
In Section~\ref{section:fact-related-work}, we discuss related work, specifically other courses about responsible AI. 
In Section~\ref{section:reproducibility}, we detail ongoing reproducibility efforts in the AI community. 
In Section~\ref{section:learning-objectives}, we explain the learning objectives for our course, and explain how we realized those objectives in Section~\ref{section:coursesetup}. 
We reflect on the feedback we received about the course in Section~\ref{section:feedback}, as well as what worked (Section~\ref{section:whatworked}) and what did not (Section~\ref{section:whatdidnt}), before concluding in Section~\ref{section:fact-conclusion}. 
