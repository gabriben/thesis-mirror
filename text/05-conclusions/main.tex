% !TEX root = ../thesis-main.tex

\chapter{Conclusions}
\label{chapter:conclusions}

\acresetall
In this thesis...


\section{Main Findings}
\label{section:conclusion-findings}
In this section, we describe our main findings across the three parts of the thesis. 

The first part of this thesis focused on ...
In Chapter~\ref{chapter:research-recfusion}, we asked our first research question:

\begin{description}\item[\acs{rq:recfusion}]\acl{rq:recfusion}\end{description}

\noindent
The answer to \textbf{RQ1} is yes: 

In Chapter~\ref{chapter:research-sigmoidf1}, we then turned to our next research question:

\begin{description}\item[\acs{rq:sigmoidf1}]\acl{rq:sigmoidf1}\end{description}

\noindent
The answer to \textbf{RQ2} is yes:

In Chapter~\ref{chapter:research-intent}, we investigated the following research question:

\begin{description}\item[\acs{rq:intent}]\acl{rq:intent}\end{description}

\noindent
The answer to \textbf{RQ3} is yes: 

We asked our final research question in Chapter~\ref{chapter:research-radio}:

\begin{description}\item[\acs{rq:radio}]\acl{rq:radio}\end{description}

\noindent
We answered \textbf{RQ4} by ...

\section{Future Directions}
\label{section:conclusion-futurework}
In this section, we describe some limitations of the methods proposed in this thesis and identify potential avenues for future work. 

\pagebreak

\subsubsection{Final thoughts} 






%%% Local Variables:
%%% mode: latex
%%% TeX-master: "../thesis-main"
%%% End:
