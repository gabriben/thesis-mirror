% !TEX root = ../thesis-main.tex

\chapter{Conclusions}
\label{chapter:conclusions}

\acresetall
In this thesis...


\section{Main Findings}
\label{section:conclusion-findings}
In this section, we describe our main findings across the three parts of the thesis. 

The first part of this thesis focused on a generative personalization pipeline throughout the user journey on a video streaming platform. Along this pipeline, we first present RecFusion, a system that uses diffusion models to generate novel and relevant recommendations for users, as part of the emerging field of Generative Information Retrieval. To make these recommendations more appealing, we also propose a method to generate personalized stills from movies using sigmoidF1, a technique that adapts the image quality and style to the user’s taste. We analyze how the user interactions on streaming platforms are influenced by not only the explicit data that is collected by web analytics, but also the implicit data that is hidden from them, using our intent-satisfaction framework. Finally, we ensure that the recommendations we generate respect the normative diversity of the users and the content providers, using RADio, a framework that measures and optimizes the fairness and diversity of the recommendations.\\

In Chapter~\ref{chapter:research-recfusion}, we asked our first research question:

\begin{description}\item[\acs{rq:recfusion}]\acl{rq:recfusion}\end{description}

\noindent
The answer to \textbf{RQ1} is yes: 

In Chapter~\ref{chapter:research-sigmoidf1}, we then turned to our next research question:

\begin{description}\item[\acs{rq:sigmoidf1}]\acl{rq:sigmoidf1}\end{description}

\noindent
The answer to \textbf{RQ2} is yes:

In Chapter~\ref{chapter:research-intent}, we investigated the following research question:

\begin{description}\item[\acs{rq:intent}]\acl{rq:intent}\end{description}

\noindent
The answer to \textbf{RQ3} is yes: 

We asked our final research question in Chapter~\ref{chapter:research-radio}:

\begin{description}\item[\acs{rq:radio}]\acl{rq:radio}\end{description}

\noindent
We answered \textbf{RQ4} by ...

\section{Future Directions}
\label{section:conclusion-futurework}

From a personalization pipeline to a personalization loop.

This was the last step of the personalization pipeline. But the implementation should not stop there. While half of the pipeline consists of active nudges and the other half of passive observation, we think that the pipeline should .

Throughout the thesis, we were motivate to pursue end-to-end solutions for each chapter. \gab{explain what I mean for each chapter}. Beyond that, we can look at the thesis as a whole as an end-to-end solutions. Generative IR also has that end-to-end aspect.

In this section, we describe some limitations of the methods proposed in this thesis and identify potential avenues for future work. 

\pagebreak

\subsubsection{Final thoughts} 






%%% Local Variables:
%%% mode: latex
%%% TeX-master: "../thesis-main"
%%% End:
