% !TEX root = ../thesis-main.tex

\acresetall

\chapter{Generative Recommendations with Diffusion}
%\chapter{Counterfactual Explanations for Graphs}
\label{chapter:research-recfusion}

\footnote[]{This chapter is under submission at International Conference on Learning Representations (ICLR) under the title ``RecFusion: A Binomial Diffusion Process for 1D Data for Recommendation'' \citep{lucic2021cfgnnexplainer}.}
\acresetall

%\todo{Add intro paragraph that connects this chapter to its RQ from chapter 1: How can we extend our explanation method for tree-based models to graph-based models? first extend problem formalization to graphs, then apply the same gradient-based optimization technique as in FOCUS}

\medskip
\noindent
\textbf{\ref{rq:recfusion}:} \acl{rq:recfusion}
\medskip

\noindent




%%!TEX root = ../main.tex

\section{Introduction}
\label{section:cfgnn-introduction}
Advances in machine learning (ML) have led to breakthroughs in several areas of science and engineering,  ranging from computer vision, to natural language processing, to conversational assistants. 
Parallel to the increased performance of ML systems, there is an increasing call for the ``understandability'' of ML models ~\citep{goebel-2018-explainable}. 
Understanding \emph{why} an ML model returns a certain output in response to a given input is important for a variety of reasons such as model debugging, aiding decison-making, or fulfilling legal requirements \citep{gdpr}. 
Having certified methods for interpreting ML predictions will help enable their use across a variety of applications~\citep{miller-2017-explanations}.



Explainable artificial intelligence (XAI) refers to the set of techniques ``\textit{focused on exposing complex AI models to humans in a systematic and interpretable manner}''~\citep{samekexplainable}. A large body of work on XAI has emerged in recent years~\citep{guidotti-2018-survey,bodria2021benchmarking}. Counterfactual explanations are used to explain predictions of individual instances in the form: ``If X had been different, Y would not have occurred''~\citep{stepin2021survey,karimi_model-agnostic_2019,schut_generating_2021}. 
Counterfactual explanations are based on counterfactual examples: modified versions of the input sample that result in an alternative output (i.e., prediction). 
If the proposed modifications are also \emph{actionable}, this is referred to as achieving recourse \citep{ustun_actionable_2019,karimi2020survey}. 

To motivate our problem, consider an ML application for computational biology: drug discovery is a task that involves generating new molecules that can be used for medicinal purposes \citep{stokes_deep_2020,xie2021mars}. 
Given a candidate molecule, a GNN can predict if this molecule has a certain property that would make it effective in treating a particular disease \citep{wieder_compact_2020,guo2021fewshot,nguyen2020metalearning}.
%If the GNN predicts it does not have this desirable property, counterfactual explanations can help identify the minimal change required in order for the molecule to have the desirable property. 
%This could help us not only generate a new molecule that has this property, but also understand the molecular structures that contribute to this property.
If the GNN predicts it does not have this desirable property, counterfactual explanations can help identify the minimal change required such that the molecule is predicted to have this property. 
This could help not only inform the design of a new molecule that has this property, but also understand the molecular structures that contribute to this property.


Although GNNs have shown state-of-the-art results on tasks involving graph data \citep{zitnik_modeling_2018,deac_drug-drug_2019}, existing methods for explaining the predictions of GNNs have primarily focused on generating subgraphs that are relevant for a particular prediction~\citep{yuan2020explainability,baldassarre_explainability_2019,duval2021graphsvx,lin_causal_2021,luo_parameterized_2020,pope_explainability_2019,schlichtkrull_interpreting_2020,vu2020pgmexplainer,ying_gnnexplainer_2019,yuan_subgraph_2021}. 
However, \emph{none of these methods are able to identify the minimal subgraph automatically} -- they all require the user to specify the size of the subgraph, $S$, in advance. 
We show that even if we adapt existing methods to the counterfactual explanation problem, and try varying values for $S$, such methods are not able to produce valid, accurate counterfactual explanations, and are therefore not well-suited to solve the counterfactual explanation problem. 
To address this gap, we propose CF-GNNExplainer, a method for generating counterfactual explanations for GNNs. 

Similar to other counterfactual methods for tabular or image data proposed in the literature~\citep{verma2020counterfactual, karimi2020survey}, CF-GNNExplainer works by perturbing input data at the instance-level. 
Unlike previous methods, CF-GNNExplainer can generate counterfactual explanations for graph data. 
In particular, our method iteratively removes edges from the original adjacency matrix based on matrix sparsification techniques, keeping track of the perturbation that leads to a change in prediction, and returning the perturbation with the smallest change w.r.t.\ the number of edges. 

\begin{figure}[t]
    \centering
    \includegraphics[width=\columnwidth]{04-research-cfgnn/images/visual.png}
    \caption{Intuition of counterfactual example generation by CF-GNNExplainer.}
    \label{fig:visual}
\end{figure}

\pagebreak

We evaluate CF-GNNExplainer on three public datasets for GNN explanations and measure its effectiveness using four metrics: fidelity, explanation size, sparsity, and accuracy. We find that CF-GNNExplainer is able to generate counterfactual examples with at least 94\% accuracy, while removing fewer than 3 edges on average. 
We make the following contributions:
\begin{enumerate}[(1)]
    \item We formalize the problem of generating counterfactual explanations for GNNs (Section~\ref{section:problem-formulation}). 
    \item We propose CF-GNNExplainer, a novel method for explaining predictions from GNNs (Section~\ref{section:cfgnn-method}). 
    \item We propose an experimental setup for holistically evaluating counterfactual explanations for GNNs (Section~\ref{section:cfgnn-experimental-setup}).
\end{enumerate}   




%%!TEX root = ../main.tex

\section{Related Work}
\label{section:focus-related-work}
Based on the taxonomy described in Chapter~\ref{chapter:introduction}, our setting in this chapter is a \emph{local explanation} problem for \emph{tree ensembles}. 
We use \emph{sensitivity analysis}, specifically counterfactual perturbations, on \emph{tabular} data to generate our explanations. 
Our work is related to counterfactual explanations in general (Section~\ref{section:focus-cf}), algorithmic recourse (Section~\ref{section:focus-recourse}), adversarial examples (Section~\ref{section:focus-adversarial}), and differentiable tree-based models (Section~\ref{section:focus-diff-trees}).

\subsection{Counterfactual Explanations}
\label{section:focus-cf}
Counterfactual examples have been used in a variety of ML areas, such as reinforcement learning \citep{madumal_explainable_2019}, deep learning \citep{alaa_deep_2017}, and XAI. 
Previous XAI methods for generating counterfactual examples are either model-agnostic \citep{poyiadzi_face_2020, karimi_model-agnostic_2019, laugel_inverse_2017, van_looveren_interpretable_2020,  mothilal_explaining_2020} or model-specific \citep{wachter_counterfactual_2017, grath_interpretable_2018, tolomei_interpretable_2017, kanamori_dace_2020, russell_efficient_2019, dhurandhar_explanations_2018}. 
Model-agnostic approaches treat the original model as a ``black-box'' and only assume query access to the model, whereas model-specific approaches typically do not make this assumption and can therefore make use of its inner workings (see Chapter~\ref{chapter:introduction}). 

Our work is a model-specific approach for generating counterfactual examples through optimization. 
Previous model-specific work for generating counterfactual examples through optimization has solely been conducted on differentiable models \citep{wachter_counterfactual_2017, grath_interpretable_2018, dhurandhar_explanations_2018}. 

\subsection{Algorithmic Recourse}
\label{section:focus-recourse}
Algorithmic recourse is a line of research that is closely related to counterfactual explanations, except that methods for algorithmic recourse include the additional restriction that the resulting explanation must be \emph{actionable} \citep{ustun_actionable_2019, joshi_towards_2019, karimi_recourse_2020, karimi_imperfect_causal_2020}. 
This is done by selecting a subset of the features to which perturbations can be applied in order to avoid explanations that suggest impossible or unrealistic changes to the feature values (i.e., change \textit{age} from \numprint{50} $\to$ \numprint{25}). 
Although this work has produced impressive theoretical results, it is unclear how realistic they are in practice, especially for complex ML models such as tree ensembles. 
Existing algorithmic recourse methods cannot solve our task because they 
\begin{enumerate*}[label=(\roman*)]
	\item are either restricted to solely linear \citep{ustun_actionable_2019} or  differentiable \citep{joshi_towards_2019} models, or
	\item  require access to causal information \citep{karimi_recourse_2020, karimi_imperfect_causal_2020}, which is rarely available in real world settings. 
\end{enumerate*}

\subsection{Adversarial Examples}
\label{section:focus-adversarial}
Adversarial examples are a type of counterfactual example with the additional constraint that the minimal perturbation results in an alternative prediction that is \emph{incorrect}. 
There are a variety of methods for generating adversarial examples \citep{goodfellow_explaining_2015,szegedy_intriguing_2014,su_one_2019,brown_adversarial_2018}; a more complete overview can be found in the work of \cite{biggio_wild_2018}. 
The main difference between adversarial examples and counterfactual examples is in the intent: adversarial examples are meant to \emph{fool} the model, whereas counterfactual examples are meant to \emph{explain} the model.


\subsection{Differentiable Tree-based Models}
\label{section:focus-diff-trees}
Part of our contribution involves constructing differentiable versions of tree ensembles by replacing each splitting threshold with a sigmoid function. 
This can be seen as using a (small) neural network to obtain a smooth approximation of each tree. 
Neural decision trees \citep{balestriero_neural_2017, yang_deep_2018} are also differentiable versions of trees, which use a full neural network instead of a simple sigmoid. 
However, these do not optimize for approximating an already trained model. Therefore, unlike our method, they are not an obvious choice for finding counterfactual examples for an existing model. 
Soft decision trees~\citep{hinton_distilling_2014} are another example of differentiable trees, which instead approximate a neural network with a decision tree. 
This can be seen as the inverse of our task. 


%%!TEX root = ../main.tex

\section{Background}
\label{section:background}
In this section, we provide background information on GNNs (Section~\ref{section:gnns-general}) and matrix sparsification (Section~\ref{section:matrix-sparsification}), both of which are necessary for understanding CF-GNNExplainer. 

\subsection{Graph Neural Networks}
\label{section:gnns-general}
Graphs are structures that represent a set of entities (nodes) and their relations (edges). 
GNNs operate on graphs to produce representations that can be used in downstream tasks such as graph or node classification. The latter is the focus of this work.
We refer to the survey papers by \citet{battaglia_relational_2018} and \citet{chami2021machine} for an overview of existing GNN methods. 

Let $f(A, X; W) \rightarrow y$ be any GNN, where $y$ is the set of possible predicted classes, $A$ is an $n \times n$ adjacency matrix, $X$ is an $n \times p$ feature matrix, and $W$ is the learned weight matrix of $f$. 
In other words, $A$ and $X$ are the inputs of $f$, and $f$ is parameterized by $W$. 

A node's representation is learned by iteratively updating the node's features based on its neighbors' features.   
The number of layers in $f$ determines which neighbors are included: if there are $\ell$ layers, then the node's final representation only includes neighbors that are at most $\ell$ hops away from that node in the graph $\graph$. 
The rest of the nodes in $\graph$ are not relevant for the computation of the node's final representation.  
We define the \emph{\cgraph{}} of a node $v$ as the set of the nodes and edges relevant for the computation of $f(v)$ (i.e., those in the $\ell$-hop neighborhood of $f$), represented as a tuple: $\compgraph = (\compadj, \compnode)$, where $\compadj$ is the subgraph adjacency matrix and $X_v$ is the node feature matrix for nodes that are at most $\ell$ hops away from $v$. We then define a node $v$ as a tuple of the form $v = (\compadj, x)$, where $x$ is the feature vector for $v$. 

\subsection{Matrix Sparsification}
\label{section:matrix-sparsification}
CF-GNNExplainer uses matrix sparsification to generate counterfactual examples, inspired by~\citet{srinivas_training_2016}, who propose a method for training sparse neural networks. 
Given a weight matrix $W$, a binary sparsification matrix is learned which is multiplied element-wise with $W$ such that some of the entries in $W$ are zeroed out. 
In the work by \citet{srinivas_training_2016}, the objective is to remove entries in the weight matrix in order to reduce the number of parameters in the model. 
In our case, we want to \emph{zero out entries in the adjacency matrix} (i.e., remove edges) in order to generate counterfactual explanations for GNNs. 
That is, we want to remove the important edges -- those that are crucial for the prediction. 
%%!TEX root = ../main.tex

\section{Problem Formulation}
\label{section:problem-formulation}
In general, a counterfactual example $\bar{x}$ for an instance $x$ according to a trained classifier $f$ is found by perturbing the features of $x$ such that $f(x) \neq f(\bar{x})$ \citep{wachter_counterfactual_2017}. 
An optimal counterfactual example $\bar{x}^*$ is one that minimizes the distance between the original instance and the counterfactual example, according to some distance function $d$. 
The resulting optimal counterfactual explanation is therefore $\Delta^*_{x} = \bar{x}^* - x$ \citep{lucic2020focus}. 

For graph data, it may not be enough to simply perturb node features, especially since they are not always available. 
This is why we are interested in generating counterfactual examples by perturbing the graph structure instead. 
In other words, we want to change the relationships between instances (i..e, nodes), rather than change the instances themselves. 
Therefore, a counterfactual example for graph data has the form $\bar{v} = (\bar{\compadj}, x)$, where $x$ is the feature vector and $\bar{\compadj}$ is a perturbed version of  $\compadj$, the adjacency matrix of the subgraph neighborhood of a node $v$. $\bar{\compadj}$ is obtained by removing some edges from $\compadj$, such that $f(v) \neq f(\bar{v})$. 
Following \citet{wachter_counterfactual_2017} and \citet{lucic2020focus}, we generate counterfactual examples by minimizing a loss function of the form:
\begin{align}
\label{eq:loss-graph}
    \mathcal{L} = \losspred(v, \bar{v} \mid f, g) + \beta \lossdist(v, \bar{v} \mid d),
\end{align}
where $v$ is the original node, $f$ is the original model, $g$ is the counterfactual model that generates $\bar{v}$, and $\losspred$ is a prediction loss that encourages $f(v) \neq f(\bar{v})$. 
$\lossdist$ is a distance loss that encourages $\bar{v}$ to be close to $v$, and $\beta$ controls how important $\lossdist$ is compared to $\losspred$. 
We want to find $\bar{v}^*$ that minimizes Equation~\ref{eq:loss-graph}: this is the optimal counterfactual example for $v$. 












%%!TEX root = ../main.tex

\section{Problem Formulation}
\label{section:focus-problem-definition}

A \emph{counterfactual explanation} for an instance $x$ and a model $f$, $\Delta_{x}$, is a minimal perturbation of $x$ that changes the prediction of $f$. 
$f$ is a probabilistic classifier, where $f(y\mid x)$ is the probability of $x$ belonging to class $y$ according to $f$.
The prediction of $f$ for $x$ is the most probable class label $y_x = \arg\max_{y} f(y \mid x)$, and
a perturbation $\bar{x}$ is a counterfactual example for $x$ if, and only if, $y_x \not = y_{\bar{x}}$, that is:
%
\begin{align}
\arg\max_{y} f(y \mid x)
\not =
\arg\max_{y'} f(y' \mid \bar{x}).
\label{eq:cfexample}
\end{align}
%
In addition to changing the prediction, the distance between $x$ and $\bar{x}$ should also be minimized. 
We therefore define an \emph{optimal counterfactual example} $\bar{x}^*$ as: 
\begin{equation}
 \bar{x}^* := \arg\min_{\bar{x}} d(x, \bar{x}) 
 \text{ such that }
y_x \not = y_{\bar{x}},
\label{eq:optimalcondition}
\end{equation}
\noindent
where $d(x, \bar{x})$ is a differentiable distance function. 
The corresponding \emph{optimal counterfactual explanation} $\Delta^*_{x}$ is:
\begin{align}
\Delta^*_{x} = \bar{x}^* - x.
\end{align} 
%\begin{align}
%\begin{split}
% \bar{x}^*&{} := \arg\min_{\bar{x}} d(x, \bar{x})  \\
%& \text{such that }
%\arg\max_{y} f(y \mid x)
%\not =
%\arg\max_{y} f(y \mid \bar{x}).
%\end{split}
%\label{eq:optimalcondition}
%\end{align}
%\todo{It would be nice to better place this in the field, i.e. cite people who agree/disagree with this definition.}
This definition aligns with previous ML work on counterfactual explanations \citep{laugel_inverse_2017, karimi_model-agnostic_2019, tolomei_interpretable_2017}. 
We note that this notion of \emph{optimality} is purely from an algorithmic perspective and does not necessarily translate to optimal changes in the real world, since the latter are dependent on the context in which they are applied. 
It should be noted that if the loss space is non-convex, it is possible that more than one optimal counterfactual explanation exists.

Minimizing the distance between $x$ and $\bar{x}$ should ensure that $\bar{x}$ is as close to the decision boundary as possible. 
This distance indicates the effort it takes to apply the perturbation in practice, and an optimal counterfactual explanation shows how a prediction can be changed with the least amount of effort.
An optimal explanation provides the user with interpretable and potentially actionable feedback related to understanding the predictions of model $f$. 


\citet{wachter_counterfactual_2017} recognized that counterfactual examples can be found through gradient descent if the task is cast as an optimization problem.
Specifically, they use a loss consisting of two components: 
\begin{enumerate*}[label=(\roman*)]
%	\item a prediction loss to change the prediction of $f$: $\mathcal{L}_{pred}(y_x, \bar{x} \mid f)$, and
	\item a prediction loss to change the prediction of $f$: $\mathcal{L}_{pred}(x, \bar{x} \mid f)$, and
	\item a distance loss to minimize the distance $d$: $\mathcal{L}_{dist}(x, \bar{x} \mid d)$.
\end{enumerate*}
The complete loss is a linear combination of these two parts, with a weight $\beta \in \mathbb{R}_{>0}$:
\begin{align}
\label{eq:mainloss}
%\mathcal{L}(x, \bar{x} \mid f, d) = \mathcal{L}_{pred}(y_x, \bar{x} \mid f) + \beta \mathcal{L}_{dist}(x, \bar{x} \mid d), 
\mathcal{L}(x, \bar{x} \mid f, d) = \mathcal{L}_{pred}(x, \bar{x} \mid f) + \beta \mathcal{L}_{dist}(x, \bar{x} \mid d).
\end{align}
%where $y_x = \arg\max_{y} f(y \mid x)$ is the original predicted class according to $f$. 
The assumption here is that an optimal counterfactual example $\bar{x}^*$ can be found by minimizing the overall loss:
\begin{align}
\bar{x}^* = \arg\min_{\bar{x}} \mathcal{L}(x, \bar{x} \mid f, d).
\end{align}
\citet{wachter_counterfactual_2017} propose a prediction loss $\mathcal{L}_{pred}$ based on the mean-squared-error. 
A clear limitation of this approach is that it assumes $f$ is differentiable.
This excludes many commonly used ML models, including tree-based models, which we focus on in this work.


\section{Method: FOCUS}
\label{section:focus-method}
To mimic many real-world scenarios, we assume there exists a trained model $f$ that we need to explain. The goal here is not to create a new, inherently interpretable tree-based model, but rather to explain a model that already exists. 


\subsection{Loss Function Definitions}

We use a hinge-loss since we assume a classification task:
%
\begin{align}
\mathcal{L}_{pred}(x, \bar{x} \mid f) = {\mathbbm{1}}\left[\arg\max_{y} f(y \mid x) = \arg\max_{y'} f(y' \mid \bar{x})\right] \cdot  f(y' \mid \bar{x}).
\end{align}
%
Allowing for flexibility in the choice of distance function allows us to tailor the explanations to the end-users' needs. We make the preferred notion of \emph{minimality} explicit through the choice of distance function. 
Given a differentiable distance function $d$, the distance loss is: 
%
\begin{align}
\mathcal{L}_{dist}(x, \bar{x}) = d(x, \bar{x}). 
\end{align}
%
Building off of \citet{wachter_counterfactual_2017}, we propose incorporating differentiable approximations of non-differentiable models to use in the gradient-based optimization framework. 
Since the approximation $\tilde{f}$ is derived from the original model $f$, it should match $f$ closely: $\tilde{f}(y \mid x) \approx f(y \mid x)$. 
We define the approximate prediction loss as follows:
\begin{align}
\widetilde{\mathcal{L}}_{pred}(x, \bar{x} \mid f, \tilde{f}) = \mathbbm{1}\left[\arg\max_{y} f(y \mid x) = \arg\max_{y'} f(y' \mid \bar{x})\right] \cdot  \tilde{f}(y' \mid \bar{x}).
\end{align}
This loss is based both on the original model $f$ and the approximation $\tilde{f}$:
the loss is active as long as the prediction according to $f$ has not changed, but its gradient is based on the differentiable $\tilde{f}$. 
This prediction loss encourages the perturbation to have a different prediction than the original instance by penalizing an unchanged instance. 
The approximation of the complete loss becomes:
\begin{equation}
\widetilde{\mathcal{L}}(x, \bar{x} \mid f, \tilde{f}, d) =\widetilde{\mathcal{L}}_{pred}(x, \bar{x} \mid f, \tilde{f}) + \beta \cdot \mathcal{L}_{dist}(x, \bar{x} \mid d).
\label{eq:approxloss}
\end{equation}
Since we assume that it approximates the complete loss, 
\begin{align}
\widetilde{\mathcal{L}}(x, \bar{x} \mid f, \tilde{f}, d) \approx \mathcal{L}(x, \bar{x} \mid f, d),
\end{align}
we also assume that an optimal counterfactual example can be found by minimizing it:
%
\begin{align}
\bar{x}^* \approx \arg\min_{\bar{x}} \, \widetilde{\mathcal{L}}(x, \bar{x} \mid f, \tilde{f}, d).
\label{eq:xbar}
\end{align}

\begin{figure}[t]
\centering
\includegraphics[scale=0.7]{04-research-focus/figures/real_tree} 
\quad
\includegraphics[scale=0.7]{04-research-focus/figures/approx_tree}
\vspace{2mm}
\caption{
Left: A decision tree $\mathcal{T}$ and node activations for a single instance. Right: a differentiable approximation of the same tree $\widetilde{\mathcal{T}}$ and activations for the same instance.
}
\label{fig:exampletrees}
\end{figure}


\subsection{Tree-based Models}
To obtain the differentiable approximation $\tilde{f}$ of $f$, we construct a probabilistic approximation of the original tree ensemble $f$.
Tree ensembles are based on decision trees; a single decision tree $\mathcal{T}$ uses a binary-tree structure to make predictions about an instance $x$ based on its features.
Figure~\ref{fig:exampletrees} shows a simple decision tree consisting of five nodes.
A node $j$ is activated if its parent node $p_j$ is activated and feature $x_{f_j}$ is on the correct side of the threshold $\theta_j$; which side is the correct side depends on whether $j$ is a \emph{left} or \emph{right} child. 
The root note is an exception, it is always activated.
Let $t_j(x)$ indicate if node $j$ is activated:
\begin{equation}
\mbox{}\hspace*{-2mm}t_j(x) =
    \begin{cases}
   1, & \text{if $j$ is the root}, \\
   t_{p_j}(x) \cdot  \mathbbm{1}[x_{f_j} > \theta_j], & \text{if $j$ is a left child}, \\
    t_{p_j}(x) \cdot  \mathbbm{1}[x_{f_j} \leq \theta_j], &\text{if $j$ is a right child}.
    \end{cases}
\end{equation}
$\forall x, \,  t_0(x) = 1$.
Nodes that have no children are called \emph{leaf nodes}; an instance $x$ always ends up in a single leaf node.
Every leaf node $j$ has its own predicted distribution $\mathcal{T}(y \mid j)$; the prediction of the full tree is given by its activated leaf node. 
Let $\mathcal{T}_{\textit{leaf}}$ be the set of leaf nodes in $\mathcal{T}$, then:
\begin{equation}
(j \in \mathcal{T}_{\textit{leaf}} \land t_j(x) = 1) \rightarrow \mathcal{T}(y \mid x) = \mathcal{T}(y \mid j).
\end{equation}
Alternatively, we can reformulate this as a sum over leaves:
\begin{equation}
\mathcal{T}(y \mid x) = \sum_{j \in \mathcal{T}_\mathit{leaf}}  t_j(x) \cdot \mathcal{T}(y \mid j).
\end{equation}
Generally, tree ensembles are deterministic; let $f$ be an ensemble of $M$ many trees with weights $\omega_m \in \mathbb{R}$, then:
\begin{equation}
f(y \mid x)
=  \arg\max_{y'} \sum_{m=1}^M \omega_m \cdot \mathcal{T}_m(y' \mid x).
\end{equation}


\subsection{Approximations of Tree-based Models}
If $f$ is not differentiable, we are unable to calculate its gradient with respect to the input $x$. 
However, the non-differentiable operations in our formulation are 
\begin{enumerate*}[label=(\roman*)]
	\item the indicator function, and
	\item a maximum operation, 
\end{enumerate*}
both of which can be approximated by differentiable functions.
First, we introduce the $\widetilde{t}_j(x)$ function that \emph{approximates the activation of node} $j$: $\widetilde{t}_j(x) \approx t_j(x)$, using a sigmoid function with parameter $\sigma \in \mathbb{R}_{>0}$:
$
\textit{sig}(z) = (1 + \exp(\sigma \cdot z))^{-1}
$
and
\begin{align}
\widetilde{t}_j(x) &{} =
    \begin{cases}
    1, & \text{if $j$ is the root}, \\
   \widetilde{t}_{p_j}(x) \cdot \textit{sig}(\theta_j {-} x_{f_j}), & \text{if $j$ is left child}, \\
   \widetilde{t}_{p_j}(x) \cdot  \textit{sig}( x_{f_j} {-} \theta_j), & \text{if $j$ is right child}.
    \end{cases}
\label{eq:sigma}    
\end{align}
As $\sigma$ increases, $\widetilde{t}_j$ approximates $t_j$ more closely.
Next, we introduce a \emph{tree approximation}:
\begin{equation}
\widetilde{\mathcal{T}}(y \mid x) = \sum_{j \in \mathcal{T}_\mathit{leaf}}  \widetilde{t}_j(x) \cdot \mathcal{T}(y \mid j).
\end{equation}
The approximation $\widetilde{\mathcal{T}}$ uses the same tree structure and thresholds as $\mathcal{T}$.
However, its activations are no longer deterministic but instead are dependent on the distance between the feature values $x_{f_j}$ and the thresholds $\theta_j$.
Lastly, we replace the maximum operation of $f$ by a softmax with temperature $\tau\in\mathbb{R}_{>0}$, resulting in:
\begin{align}
\tilde{f}(y \mid x)
= \frac{
\exp\left(\tau \cdot \sum_{m=1}^M \omega_m \cdot \widetilde{\mathcal{T}}_m(y \mid x)\right)
}{
\sum_{y'} \exp\left(\tau \cdot \sum_{m=1}^M \omega_m \cdot \widetilde{\mathcal{T}}_m(y' \mid x)\right)
}.
\label{eq:tau}
\end{align}
The approximation $\tilde{f}$ is based on the original model $f$ and the parameters $\sigma$ and $\tau$.
This approximation is applicable to any tree-based model, and 
how well $\tilde{f}$ approximates $f$ depends on the choice of $\sigma$ and $\tau$.
The approximation is potentially perfect since
\begin{align}
\lim_{\sigma,\tau\rightarrow\infty}
\tilde{f}(y \mid x) = f(y \mid x).
\end{align}
%
\subsection{Our Method: FOCUS}
We call our method FOCUS: Flexible Optimizable CounterfactUal Explanations for Tree EnsembleS. 
It takes as input an instance $x$, a tree-based classifier $f$, and two hyperparameters: $\sigma$ and $\tau$, which we use to create the approximation $\tilde{f}$. 
Following Equation~\ref{eq:xbar}, FOCUS outputs the optimal counterfactual example $\bar{x}^*$, from which we derive the optimal counterfactual explanation $\Delta^*_{x} = \bar{x}^* - x$. 
%\mdr{Now introduce the FOCUS acronym and say what the method is, including what the inputs and outputs of the method are.}

\subsection{Effects of Hyperparameters}
Increasing $\sigma$ in $\tilde{f}$ eventually leads to exact approximations of the indicator functions, while increasing $\tau$ in $\tilde{f}$ leads to a completely unimodal softmax distribution. 
It should be noted that our approximation $\tilde{f}$ is not intended to replace the original model $f$ but rather to create a differentiable version of $f$ from which we can generate counterfactual examples through optimization. 
In practice, the original model $f$ would still be used to make predictions and the approximation would solely be used to generate counterfactual examples. 

%%!TEX root = ../main.tex

\section{Experimental Setup}
\label{section:cfgnn-experimental-setup}

In this section, we outline our experimental setup for evaluating CF-GNNExplainer, including the datasets and models used (Section~\ref{section:cfgnn-datasets}), the baselines we compare against (Section~\ref{section:baselines}), the evaluation metrics (Section~\ref{section:cfgnn-metrics}), and the hyperparameter search method (Section~\ref{section:hyperparams}). 
In total, we run approximately 375 hours of experiments on one Nvidia TitanX Pascal GPU with access to 12GB RAM. 


\subsection{Datasets and Models}
\label{section:cfgnn-datasets}
Given the challenges associated with defining and evaluating the accuracy of XAI methods~\citep{doshi-2017-towards}, we first focus on synthetic tasks where we know the ground-truth explanations. 
Although there exist real graph classification datasets with ground-truth explanations~\citep{mutag_dataset}, there do not exist any real node classification datasets with ground-truth explanations, which is the task we focus on in this chapter. 
Building such a dataset would be an excellent contribution, but is outside the scope of this paper.

In our experiments, we use the \synfour{}, \textsc{tree-grids}, \synone{} datasets from the work by \citet{ying_gnnexplainer_2019}. 
These datasets were created specifically for the task of explaining node classification predictions from GNNs. 
Each dataset consists of (i) a base graph, (ii) motifs that are attached to random nodes of the base graph, and (iii) additional edges that are randomly added to the overall graph. 
They are all undirected graphs. 
The classification task is to determine whether or not the nodes are part of the motif. 
The purpose of these datasets is to have a ground-truth for the ``correctness'' of an explanation: for nodes in the motifs, the explanation is the motif itself \citep{luo_parameterized_2020}. 
The dataset statistics are available in Table~\ref{table:stats}. 





\synfour{} consists of a binary tree base graph with 6-cycle motifs, \textsc{tree-grids} also has a binary tree as its base graph, but with 3$\times$3 grids as the motifs. 
For \synone{}, the base graph is a Barabasi-Albert (BA) graph with house-shaped motifs, where each motif consists of 5 nodes (one for the top of the house, two in the middle, and two on the bottom). 
Here, there are four possible classes (not in motif, in motif: top, middle, bottom). 
We note that compared to the other two datasets, the \synone{} dataset is much more densely connected -- the node degree is more than twice as high as that of the \synfour{} or \synfive{} datasets, and the average number of nodes and edges in each node's computation graph is order(s) of magnitude larger. 
We use the same experimental setup (i.e., dataset splits, model architecture) as \citet{ying_gnnexplainer_2019} to train a 3-layer GCN (hidden size = 20) for each task. 
Our GCNs have at least 87\% accuracy on the test set. 






\begin{table}[]
\caption{Dataset statistics. The \# edges in the motif indicates the size of the ground truth (GT) explanation. }
%\todo{Double check notation: use commas for bigger numbers or not.}		% NO commas!
\label{table:stats}
\centering
\begin{tabular}{lrrr}
\toprule
                          & \multicolumn{1}{c}{\textsc{Tree}}   & \multicolumn{1}{c}{\textsc{Tree}} & \multicolumn{1}{c}{\textsc{BA}}     \\
                          & \multicolumn{1}{c}{\textsc{Cycles}} & \multicolumn{1}{c}{\textsc{Grid}} & \multicolumn{1}{c}{\textsc{Shapes}} \\ 
\midrule
\# classes                 & 2                         & 2               & 4                          \\ 

\# nodes in motif                 & 6                        & 9            & 5                        \\
\# edges in motif      (GT)            & 6                       & 12             & 6                \\
\midrule
\# nodes in total                  & 871                        & 1231            & 700                        \\
\# edges in total                  & 1950                       & 3410             & 4100                \\

\midrule
Avg node degree           & 2.27                       & 2.77                     & 5.87                       \\
Avg \# nodes in $\compadj$ & 19.12                      & 30.69                    & 304.40                     \\
Avg \# edges in $\compadj$ & 18.99                      & 33.94                    & 1106.24                    \\
\bottomrule
\end{tabular}
\end{table}




\subsection{Baselines}
\label{section:baselines}
Since existing GNN XAI methods give explanations in the form of relevant subgraphs as opposed to counterfactual examples, it is not straightforward to identify baselines for our experiments that ensure a fair comparison between methods. 
To evaluate CF-GNNExplainer, we compare against 4 baselines: \baserand{}, \basekeep{}, \baserm{}, and \gnnexplainer{}.
The random perturbation is meant as a sanity check. 
We randomly initialize the entries of $\perturbl \in \left[-1, 1\right]$ and apply the same sigmoid transformation and thresholding as described in Section~\ref{section:adjacency-perturb}. 
We repeat this $K$ times and keep track of the most minimal perturbation resulting in a counterfactual example. 
Next, we compare against baselines that are based on the ego graph of $v$ (i.e., its 1-hop neighbourhood): \basekeep{} keeps all edges in the ego graph of $v$, while \baserm{} removes all edges in the ego graph of $v$. 

Our fourth baseline is based on \gnnexplainer{} by \citet{ying_gnnexplainer_2019}, which identifies the $S$ most relevant edges for the prediction (i.e., the most relevant subgraph of size $S$). 
To generate counterfactual explanations, we remove the subgraph generated by \gnnexplainer{}. 
We include this method in our experiments in order to have a baseline based on a prominent GNN XAI method, but we note that subgraph-retrieving methods like \gnnexplainer{} are not meant for generating counterfactual explanations. 
Unlike our method, \gnnexplainer{} cannot automatically find a \emph{minimal} subgraph and therefore requires the user to determine the number of edges to keep in advance (i.e., the value of $S$). 
As a result, we cannot evaluate how minimal its counterfactual explanations are, but we can compare it against our method in terms of 
\begin{inparaenum}[(i)]
	\item its ability to generate valid counterfactual examples, and 
	\item how accurate those counterfactual examples are.
\end{inparaenum}
We report results on \gnnexplainer{} for $ S \in \{1, 2, 3, 4, 5,$ GT$\}$, where GT is the size of the ground truth explanation (i.e., the number of edges in the motif, see Table~\ref{table:stats}). 




\subsection{Metrics}
\label{section:cfgnn-metrics}
We generate a counterfactual example for each node in the graph separately and evaluate in terms of four metrics.

\medskip \noindent
\textbf{Fidelity:} is defined as the proportion of nodes where the original predictions match the prediction for the explanations \citep{molnar2019,ribeiro-2016-should}. Since we generate counterfactual examples, we do not want the original prediction to match the prediction for the explanation, so we want a low value for fidelity. 

\medskip \noindent
\textbf{Explanation Size:} is the number of removed edges. It corresponds to the $\lossdist$ term in Equation~\ref{eq:loss-graph}: the difference between the original $\compadj$ and the counterfactual $\bar{\compadj}$. Since we want to have \emph{minimal} explanations, we want a small value for this metric. Note that we cannot evaluate this metric for \gnnexplainer{}. 

\medskip \noindent
\textbf{Sparsity:} measures the proportion of edges in $\compadj$ that are removed \citep{yuan2020explainability}. A value of 0 indicates all edges in $\compadj$ were removed. Since we want \emph{minimal} explanations, we want a value close to 1. Note that we cannot evaluate this metric for \gnnexplainer{}.

\medskip \noindent
\textbf{Accuracy:} is the mean proportion of explanations that are ``correct''. Following the work by \citet{ying_gnnexplainer_2019, luo_parameterized_2020}, we only compute accuracy for nodes that are originally predicted as being part of the motifs, since accuracy can only be computed on instances for which we know the ground truth explanations. 
Given that we want \emph{minimal} explanations, we consider an explanation to be correct if it \emph{exclusively} involves edges that are inside the motifs (i.e., only removes edges that are within the motifs). 

%The exact calculations of all metrics can be found in the public code base at \url{https://github.com/cf-gnnexplainer}. 



\subsection{Hyperparameter Search}
\label{section:hyperparams}
We experiment with different optimizers and hyperparameter values for the number of iterations $K$, the trade-off parameter $\beta$, the learning rate $\alpha$, and the Nesterov momentum $m$ (when applicable). 
We choose the setting that produces the most counterfactual examples. 
We test the number of iterations $K \in \{100, 300, 500\}$, the trade-off parameter $\beta \in \{0.1, 0.5\}$, the learning rate $\alpha \in \{0.005, 0.01, 0.1, 1\}$, and the Nesterov momentum $m \in \{0, 0.5, 0.7, 0.9\}$. 
We test Adam, SGD and AdaDelta as optimizers. 
We find that for all three datasets, the SGD optimizer gives the best results, with $k = 500$, $\beta = 0.5$, and $\alpha = 0.1$. 
For the \synfour{} and \synfive{} datasets, we set $m = 0$, while for the \synone{} dataset, we use $m = 0.9$. 




%%!TEX root = ../main.tex

\section{Results}
\label{section:cfgnn-results}


We evaluate CF-GNNExplainer in terms of the metrics outlined in Section~\ref{section:cfgnn-metrics}. 
The results are shown in Table~\ref{table:results1} and Table~\ref{table:results-gnnexplainer}.  
In cases where the baselines outperform CF-GNNExplainer on a particular metric, they perform poorly on the rest of the metrics, or on other datasets. 




\subsection{Main Findings}

\textbf{Fidelity:}
CF-GNNExplainer outperforms \basekeep{} across all three datasets, and outperforms \baserm{} for \synfour{} and \synfive{} in terms of fidelity. 
We find that \baserand{} has the lowest fidelity in all cases -- it is able to find counterfactual examples for every single node. 
In the following subsections, we will see that this corresponds to poor performance on the other metrics.

\begin{table*}[h]
\centering
\caption{Results comparing our method (abbreviated as \OurShort{}) to \baserand{}, \basekeep{}, and \baserm{}. Below each metric, $\blacktriangledown$ indicates a low value is desirable, while $\blacktriangle$ indicates a high value is desirable.}
\label{table:results1}
\setlength{\tabcolsep}{4pt}
\scriptsize{
\begin{tabular}{lrrrr rrrr rrrr}
\toprule
\multicolumn{1}{c}{} & \multicolumn{4}{c}{\synfour{}}                                                                                                                 & \multicolumn{4}{c}{\synfive{}}                                                                                                                   & \multicolumn{4}{c}{\synone{}}                                                                                                                  \\ 
\cmidrule(r){2-5}\cmidrule(r){6-9}\cmidrule{10-13} 
               & \multicolumn{1}{c}{Fid.} & \multicolumn{1}{c}{Size} & \multicolumn{1}{c}{Spar.} & \multicolumn{1}{c}{Acc.} & \multicolumn{1}{c}{Fid.} & \multicolumn{1}{c}{Size} & \multicolumn{1}{c}{Spar.} & \multicolumn{1}{c}{Acc.} & \multicolumn{1}{c}{Fid.} & \multicolumn{1}{c}{Size} & \multicolumn{1}{c}{Spar.} & \multicolumn{1}{c}{Acc.} \\

% & \multicolumn{1}{c}{$\downarrow$} &\multicolumn{1}{c}{$\downarrow$} &\multicolumn{1}{c}{$\uparrow$} & \multicolumn{1}{c}{$\uparrow$} & \multicolumn{1}{c}{$\downarrow$} &\multicolumn{1}{c}{$\downarrow$} &\multicolumn{1}{c}{$\uparrow$} & \multicolumn{1}{c}{$\uparrow$} & \multicolumn{1}{c}{$\downarrow$} &\multicolumn{1}{c}{$\downarrow$} &\multicolumn{1}{c}{$\uparrow$} & \multicolumn{1}{c}{$\uparrow$} \\

Method & \multicolumn{1}{c}{$\blacktriangledown$} &\multicolumn{1}{c}{$\blacktriangledown$} &\multicolumn{1}{c}{$\blacktriangle$} & \multicolumn{1}{c}{$\blacktriangle$} & \multicolumn{1}{c}{$\blacktriangledown$} &\multicolumn{1}{c}{$\blacktriangledown$} &\multicolumn{1}{c}{$\blacktriangle$} & \multicolumn{1}{c}{$\blacktriangle$} & \multicolumn{1}{c}{$\blacktriangledown$} &\multicolumn{1}{c}{$\blacktriangledown$} &\multicolumn{1}{c}{$\blacktriangle$} & \multicolumn{1}{c}{$\blacktriangle$} \\
\midrule
\baserand{}               & \textbf{0.00}                     & 4.70                              & 0.79                                & 0.63                               & \textbf{0.00}                     & 9.06                              & 0.75                                & 0.77                               & \textbf{0.00}                     & 503.31                            & 0.58                                & 0.17                              \\
\basekeep{}                 & 0.32                              & 15.64                             & 0.13                                & 0.45                               & 0.32                              & 29.30                             & 0.09                                & 0.72                               & 0.60                              & 504.18                            & 0.05                                & 0.18                              \\
\baserm{}              & 0.46                              & 2.11                              & 0.89                                & ---                                  & 0.61                              & 2.27                              & 0.92                                & ---                                  & 0.21                              & 10.56                             & 0.97                                & \textbf{0.99}                     \\


% \gnnexpshort{} ($S=$ GT) &  0.55 &	6.00 & 0.57 &	0.46 &	0.35 &	11.83 &	0.53 &	0.74 &	0.82 &	6.00 &	0.79 &	0.24    \\

\midrule
CF-GNN              & 0.21                              & \textbf{2.09}                     & \textbf{0.90}                       & \textbf{0.94}                      & 0.07                              & \textbf{1.47}                     & \textbf{0.94}                       & \textbf{0.96}                      & 0.39                              & \textbf{2.39}                     & \textbf{0.99}                       & 0.96                 \\
\bottomrule
\end{tabular}
}
\end{table*}





\medskip \noindent
\textbf{Explanation Size:}
Figures~\ref{fig:random-explanation-size} to~\ref{fig:explanation-size} show histograms of the explanation size for CF-GNNExplainer and the baselines. 
We see that across all three datasets, CF-GNNExplainer has the smallest (i.e., most minimal) explanation sizes. 
This is especially true when comparing to \baserand{} and \basekeep{} for the \synone{} dataset, where we had to use a different scale for the $x$-axis due to how different the explanation sizes were. 
We postulate that this difference could be because \synone{} is a much more densely connected graph;
it has fewer nodes but more edges compared to the other two datasets, and the average number of nodes and edges in the \cgraph{} is order(s) of magnitude larger (see Table~\ref{table:stats}). 
Therefore, when performing random perturbations, there is substantial opportunity to remove edges that do not necessarily need to be removed, leading to much larger explanation sizes.
When there are many edges in the \cgraph{}, removing everything except the 1-hop neighbourhood, as is done in \basekeep{}, also results in large explanation sizes. 
In contrast, the loss function used by CF-GNNExplainer ensures that only a few edges are removed, which is the desirable behavior since we want minimal explanations. 

\pagebreak

\medskip \noindent
\textbf{Sparsity:}
CF-GNNExplainer outperforms the \baserand{}, \baserm{}, \basekeep{} baselines for all three datasets in terms of sparsity.
%\footnote{GNNExplainer cannot be evaluated on sparsity.} 
We note CF-GNNExplainer and \baserm{} perform much better on this metric in comparison to the other methods, which aligns with the results from explanation size. 

\medskip \noindent
\textbf{Accuracy:}
We observe that CF-GNNExplainer has the highest accuracy for the \synfour{} and \synfive{} datasets, whereas \baserm{} has the highest accuracy for \synone{}. 
However, we are unable to calculate the accuracy of \baserm{} for the other two datasets since it is unable to generate \emph{any} counterfactual examples for motif nodes, contributing to the low sparsity on those datasets. 
We observe accuracy levels upwards of 94\% for CF-GNNExplainer across \emph{all} datasets, indicating that it is consistent in correctly removing edges that are crucial for the initial predictions in the vast majority of cases (see Table~\ref{table:results1}). 






\begin{figure*}[]

    \centering

    \includegraphics[scale=0.27]{04-research-cfgnn/images/tree-cycles-random.png}
    \includegraphics[scale=0.27]{04-research-cfgnn/images/tree-grid-random.png}
    \includegraphics[scale=0.27]{04-research-cfgnn/images/ba-shapes-random.png}
    
        \caption{Histograms showing the proportion of counterfactual examples that have a certain explanation size from \baserand{}. Note the $x$-axis for \synone{} goes up to 1500. Left: \synfour{}, Middle: \synfive{}, Right: \synone{}.  }
        \label{fig:random-explanation-size}
        \bigskip \bigskip
        
    \includegraphics[scale=0.27]{04-research-cfgnn/images/tree-cycles-keep.png}
    \includegraphics[scale=0.27]{04-research-cfgnn/images/tree-grid-keep.png}
    \includegraphics[scale=0.27]{04-research-cfgnn/images/ba-shapes-keep.png}
    
        \caption{Histograms showing the proportion of counterfactual examples that have a certain explanation size from \basekeep{}. Note the $x$-axis for \synone{} goes up to 1500. Left: \synfour{}, Middle: \synfive{}, Right: \synone{}. }
        \label{fig:keep-explanation-size}
        \bigskip \bigskip
        
        
    \includegraphics[scale=0.27]{04-research-cfgnn/images/tree-cycles-remove.png}
    \includegraphics[scale=0.27]{04-research-cfgnn/images/tree-grid-remove.png}
    \includegraphics[scale=0.27]{04-research-cfgnn/images/ba-shapes-remove.png}
    
        \caption{Histograms showing the proportion of counterfactual examples that have a certain explanation size from \baserm{}. Note the $x$-axis for \synone{} goes up to 70. Left: \synfour{}, Middle: \synfive{}, Right: \synone{}. }
        \label{fig:remove-explanation-size}
        \bigskip \bigskip
        
    % \includegraphics[scale=0.38]{images/tree-cycles-gnnexplainer.png}
    % \includegraphics[scale=0.38]{images/tree-grid-gnnexplainer.png}
    % \includegraphics[scale=0.38]{images/ba-shapes-gnnexplainer.png}
    
    %     \caption{Histograms showing explanation size from \gnnexplainer{} for $S=$ GT. Note that the y-axis goes up to 1. Left: \synfour{}, Middle: \synfive{}, Right: \synone{}.}
    %     \label{fig:gnnexplainer-explanation-size}

    \includegraphics[scale=0.27]{04-research-cfgnn/images/tree-cycles.png}
    \includegraphics[scale=0.27]{04-research-cfgnn/images/tree-grid.png}
    \includegraphics[scale=0.27]{04-research-cfgnn/images/ba-shapes.png}
    
        \caption{Histograms showing the proportion of counterfactual examples that have a certain explanation size from CF-GNNExplainer. Note the $x$-axis for \synone{} goes up to 70. Left: \synfour{}, Middle: \synfive{}, Right: \synone{}. }
        \label{fig:explanation-size}
        
\end{figure*}







\subsection{Comparison to \gnnexplainer{}}
Table~\ref{table:results-gnnexplainer} shows the results comparing our method to \gnnexplainer{}. We find that our method outperforms \gnnexplainer{} across all three datasets in terms of both fidelity and accuracy, for all tested values of $S$. 
However, this is not surprising since \gnnexplainer{} is not meant for generating counterfactual explanations, so we cannot expect it to perform well on a task it was not designed for. 
We cannot compare explanation size or sparsity fairly since \gnnexplainer{} requires the user to input the value of $S$. 



\begin{table*}[]
\centering
\caption{Results comparing our method to \gnnexplainer{}. \gnnexplainer{} cannot find $S$ automatically, so we try varying values of $S$. GT indicates the size of the ground truth explanation for each dataset. CF-GNNExplainer finds $S$ automatically. Below each metric, $\blacktriangledown$ indicates a low value is desirable, while $\blacktriangle$ indicates a high value is desirable.}
\label{table:results-gnnexplainer}
\setlength{\tabcolsep}{4pt}
\scriptsize{
\begin{tabular}{lrrrr rrrr rrrr}
\toprule
\multicolumn{1}{c}{} & \multicolumn{4}{c}{\synfour{}}                                                                                                                 & \multicolumn{4}{c}{\synfive{}}                                                                                                                   & \multicolumn{4}{c}{\synone{}}                                                                                                                  \\ 
\cmidrule(r){2-5}\cmidrule(r){6-9}\cmidrule{10-13} 
               & \multicolumn{1}{c}{Fid.} & \multicolumn{1}{c}{Size} & \multicolumn{1}{c}{Spars.} & \multicolumn{1}{c}{Acc.} & \multicolumn{1}{c}{Fid.} & \multicolumn{1}{c}{Size} & \multicolumn{1}{c}{Spars.} & \multicolumn{1}{c}{Acc.} & \multicolumn{1}{c}{Fid.} & \multicolumn{1}{c}{Size} & \multicolumn{1}{c}{Spars.} & \multicolumn{1}{c}{Acc.} \\
\gnnexpshort{} & \multicolumn{1}{c}{$\blacktriangledown$} &\multicolumn{1}{c}{$\blacktriangledown$} &\multicolumn{1}{c}{$\blacktriangle$} & \multicolumn{1}{c}{$\blacktriangle$} & \multicolumn{1}{c}{$\blacktriangledown$} &\multicolumn{1}{c}{$\blacktriangledown$} &\multicolumn{1}{c}{$\blacktriangle$} & \multicolumn{1}{c}{$\blacktriangle$} & \multicolumn{1}{c}{$\blacktriangledown$} &\multicolumn{1}{c}{$\blacktriangledown$} &\multicolumn{1}{c}{$\blacktriangle$} & \multicolumn{1}{c}{$\blacktriangle$} \\
\midrule


 $S=1$ & 0.65 & 1.00 & 0.92 & 0.61 & 0.69 & 1.00 & 0.96 & 0.79 & 0.90 & 1.00 & 0.94 & 0.52 \\
 $S=2$ & 0.59 & 2.00 & 0.85 & 0.54 & 0.51 & 2.00 & 0.92 & 0.78 & 0.85 & 2.00 & 0.91 & 0.40  \\
 $S=3$ & 0.56 & 3.00 & 0.79 & 0.51 & 0.46 & 3.00 & 0.88 & 0.79 & 0.83 & 3.00 & 0.87 & 0.34 \\
 $S=4$ & 0.58 & 4.00 & 0.72 & 0.48 & 0.42 & 4.00 & 0.84 & 0.79 & 0.83 & 4.00 & 0.83 & 0.31 \\
 $S=5$ & 0.57 & 5.00 & 0.66 & 0.46 & 0.40 & 5.00 & 0.80  & 0.79 & 0.81 & 5.00 & 0.81 & 0.27 \\
 $S=$ GT &  0.55 &	6.00 & 0.57 &	0.46 &	0.35 &	11.83 &	0.53 &	0.74 &	0.82 &	6.00 &	0.79 &	0.24    \\

\midrule
CF-GNN               & \textbf{0.21}                              & 2.09                     & 0.90                       & \textbf{0.94}                      & \textbf{0.07}                              & 1.47                     & 0.94                       & \textbf{0.96}                      & \textbf{0.39}                              & 2.39                     & 0.99                       & \textbf{0.96}                 \\
\bottomrule
\end{tabular}
}
\end{table*}


\subsection{Summary of Results} 
Evaluating on four distinct metrics for each dataset gives us a more holistic view of the results. 
We find that across all three datasets, CF-GNNExplainer can generate counterfactual examples for the majority of nodes in the test set (i.e., low fidelity), while only removing a small number of edges (i.e., low explanation size, high sparsity). For nodes where we know the ground truth (i.e., those in the motifs) we achieve at least 94\% accuracy. 

Although \baserand{} can generate counterfactual examples for every node, they are not very minimal or accurate. 
The latter is also true for \basekeep{} -- in general, it has the worst scores for explanation size, sparsity and accuracy. 
\gnnexplainer{} performs at a similar level as \basekeep{}, indicating that although it is a prominent GNN XAI method, it is not well-suited for solving the counterfactual explanation problem. 

\baserm{} is competitive in terms of explanation size, but it performs poorly in terms of fidelity for the \synfour{} and \synfive{} datasets, and its accuracy on these datasets is unknown since it is unable to generate \emph{any} counterfactual examples for nodes in the motifs. 
These results show that our method is simple and effective in solving the counterfactual explanation task, unlike the baselines we test. 










%%!TEX root = ../main.tex











\section{Conclusion}
\label{section:cfgnn-conclusion}
In this chapter, we propose CF-GNNExplainer, a method for generating counterfactual explanations for any GNN. Our simple and effective method is able to generate counterfactual explanations that are (i) minimal, both in terms of the absolute number of edges removed (explanation size), as well as the proportion of the \cgraph{} that is perturbed (sparsity), and (ii) accurate, in terms of removing edges that we know to be crucial for the initial predictions. 

We evaluate our method on three commonly used datasets for GNN explanation tasks and find that these results hold across all three datasets. 
We find that existing GNN XAI methods are not well-suited to solving the counterfactual explanation task, while CF-GNNExplainer is able to reliably produce minimal, accurate counterfactual explanations. 

This answers \textbf{\ref{rq:cf-gnn}}: we can generate counterfactual explanations for graph-based models by extending the problem formalization from Chapter~\ref{chapter:research-focus} to accommodate graph data. 
We do so by introducing a perturbation matrix that acts on the adjacency matrix to remove edges in the graph, then applying similar gradient-based optimization techniques as in Chapter~\ref{chapter:research-focus} for each instance in the dataset. 
In the following chapter, we will investigate how to generate explanations that are specific to a particular real-world use case and evaluate them on real users. 


%
\section{Results Table Including Standard Deviations}
Here we show Table 2 from the manuscript including standard deviations. 
We report standard deviation for two metrics: \textit{Explanation Size} and \textit{Sparsity}, since both of these involve taking the mean over the entire dataset. 
Standard deviation does not apply to \textit{Fidelity} or \textit{Accuracy} because these metrics represent a proportion as opposed to a mean. 


\begin{table}[h]
\centering
\caption*{Table 2: Results comparing our method (denoted \OurShort{}) to \baserand{}, \basekeep{}, and \baserm{}. Below each metric, $\blacktriangledown$ indicates a low value is desirable, while $\blacktriangle$ indicates a high value is desirable.}
\label{table:results}
%\setlength{\tabcolsep}{4pt}
%\footnotesize{
\resizebox{\textwidth}{!}{\begin{tabular}{lrrrrrrrrrrrr}
\toprule
\multicolumn{1}{c}{} & \multicolumn{4}{c}{\synfour{}}                                                                                                                 & \multicolumn{4}{c}{\synfive{}}                                                                                                                   & \multicolumn{4}{c}{\synone{}}                                                                                                                  \\ 
\cmidrule(r){2-5}\cmidrule(r){6-9}\cmidrule{10-13} 
               & \multicolumn{1}{c}{\textit{Fid.}} & \multicolumn{1}{c}{\textit{Size}} & \multicolumn{1}{c}{\textit{Spars.}} & \multicolumn{1}{c}{\textit{Acc.}} & \multicolumn{1}{c}{\textit{Fid.}} & \multicolumn{1}{c}{\textit{Size}} & \multicolumn{1}{c}{\textit{Spars.}} & \multicolumn{1}{c}{\textit{Acc.}} & \multicolumn{1}{c}{\textit{Fid.}} & \multicolumn{1}{c}{\textit{Size}} & \multicolumn{1}{c}{\textit{Spars.}} & \multicolumn{1}{c}{\textit{Acc.}} \\

% & \multicolumn{1}{c}{$\downarrow$} &\multicolumn{1}{c}{$\downarrow$} &\multicolumn{1}{c}{$\uparrow$} & \multicolumn{1}{c}{$\uparrow$} & \multicolumn{1}{c}{$\downarrow$} &\multicolumn{1}{c}{$\downarrow$} &\multicolumn{1}{c}{$\uparrow$} & \multicolumn{1}{c}{$\uparrow$} & \multicolumn{1}{c}{$\downarrow$} &\multicolumn{1}{c}{$\downarrow$} &\multicolumn{1}{c}{$\uparrow$} & \multicolumn{1}{c}{$\uparrow$} \\

Method & \multicolumn{1}{c}{$\blacktriangledown$} &\multicolumn{1}{c}{$\blacktriangledown$} &\multicolumn{1}{c}{$\blacktriangle$} & \multicolumn{1}{c}{$\blacktriangle$} & \multicolumn{1}{c}{$\blacktriangledown$} &\multicolumn{1}{c}{$\blacktriangledown$} &\multicolumn{1}{c}{$\blacktriangle$} & \multicolumn{1}{c}{$\blacktriangle$} & \multicolumn{1}{c}{$\blacktriangledown$} &\multicolumn{1}{c}{$\blacktriangledown$} &\multicolumn{1}{c}{$\blacktriangle$} & \multicolumn{1}{c}{$\blacktriangle$} \\
\midrule
\baserand{}               & \textbf{0.00}                     & 4.70   $\pm$       4.28                    & 0.79     $\pm$ 0.07                           & 0.63                               & \textbf{0.00}                     & 9.06    $\pm$ 6.81                          & 0.75      $\pm$ 0.07                          & 0.77                               & \textbf{0.00}                     & 503.31   $\pm$ 332.61                         & 0.58   $\pm$ 0.10                             & 0.17                              \\
\basekeep{}                 & 0.32                              & 15.64      $\pm$ 12.36                       & 0.13       $\pm$ 0.06                         & 0.45                               & 0.32                              & 29.30      $\pm$16.53                       & 0.09      $\pm$ 0.04                          & 0.72                               & 0.60                              & 504.18    $\pm$ 567.92                        & 0.05      $\pm$ 0.05                          & 0.18                              \\
\baserm{}              & 0.46                              & 2.11          $\pm$ 1.04                    & 0.89         $\pm$ 0.04                       & ---                                  & 0.61                              & 2.27         $\pm$ 1.28                     & 0.92      $\pm$ 0.04                          & ---                                  & 0.21                              & 10.56      $\pm$ 20.11                       & 0.97     $\pm$ 0.04                           & \textbf{0.99}                     \\


% \gnnexpshort{} ($S=$ GT) &  0.55 &	6.00 & 0.57 &	0.46 &	0.35 &	11.83 &	0.53 &	0.74 &	0.82 &	6.00 &	0.79 &	0.24    \\

\midrule
\OurShort{}               & 0.21                              & \textbf{2.09}     $\pm$ 2.21                & \textbf{0.90}    $\pm$ 0.07                   & \textbf{0.94}                      & 0.07                              & \textbf{1.47}    $\pm$0.77                 & \textbf{0.94}    $\pm$ 0.04                   & \textbf{0.96}                      & 0.39                              & \textbf{2.39}    $\pm$ 1.39                 & \textbf{0.99}      $\pm$ 0.01                 & 0.96                 \\
\bottomrule
\end{tabular}}
%}
\end{table}



All differences between \OurMethod{} and the baselines are statistically significant with $\alpha=0.01$ using a $t$-test, with two exceptions: \OurMethod{} vs. \baserm{} on the \synfive{} dataset, for (i) \textit{Explanation Size} and (ii) \textit{Sparsity}. However, \OurMethod{} outperforms \baserm{} significantly on \textit{Fidelity} and \textit{Accuracy} ($\alpha=0.01$). 


We do not calculate standard deviations for Table 3, where we compare against \gnnexplainer{}, since we cannot evaluate \gnnexplainer{} on \textit{Explanation Size} or \textit{Sparsity}. This is because the user must specify the \textit{Explanation Size} in advance (see Sections 6.2, 7.2). We can only evaluate \gnnexplainer{} on \textit{Fidelity} and \textit{Accuracy}, neither of which require standard deviation calculations since they represent proportions. 





\section*{Reproducibility}
To facilitate the reproducibility of the work in this chapter, our code is available at \url{https://github.com/gabriben/recfusion}.








