%!TEX root = ../main.tex

\section{Problem Formulation}
\label{section:problem-formulation}
In general, a counterfactual example $\bar{x}$ for an instance $x$ according to a trained classifier $f$ is found by perturbing the features of $x$ such that $f(x) \neq f(\bar{x})$ \citep{wachter_counterfactual_2017}. 
An optimal counterfactual example $\bar{x}^*$ is one that minimizes the distance between the original instance and the counterfactual example, according to some distance function $d$. 
The resulting optimal counterfactual explanation is therefore $\Delta^*_{x} = \bar{x}^* - x$ \citep{lucic2020focus}. 

For graph data, it may not be enough to simply perturb node features, especially since they are not always available. 
This is why we are interested in generating counterfactual examples by perturbing the graph structure instead. 
In other words, we want to change the relationships between instances (i..e, nodes), rather than change the instances themselves. 
Therefore, a counterfactual example for graph data has the form $\bar{v} = (\bar{\compadj}, x)$, where $x$ is the feature vector and $\bar{\compadj}$ is a perturbed version of  $\compadj$, the adjacency matrix of the subgraph neighborhood of a node $v$. $\bar{\compadj}$ is obtained by removing some edges from $\compadj$, such that $f(v) \neq f(\bar{v})$. 
Following \citet{wachter_counterfactual_2017} and \citet{lucic2020focus}, we generate counterfactual examples by minimizing a loss function of the form:
\begin{align}
\label{eq:loss-graph}
    \mathcal{L} = \losspred(v, \bar{v} \mid f, g) + \beta \lossdist(v, \bar{v} \mid d),
\end{align}
where $v$ is the original node, $f$ is the original model, $g$ is the counterfactual model that generates $\bar{v}$, and $\losspred$ is a prediction loss that encourages $f(v) \neq f(\bar{v})$. 
$\lossdist$ is a distance loss that encourages $\bar{v}$ to be close to $v$, and $\beta$ controls how important $\lossdist$ is compared to $\losspred$. 
We want to find $\bar{v}^*$ that minimizes Equation~\ref{eq:loss-graph}: this is the optimal counterfactual example for $v$. 











