% !TEX root = thesis-main.tex
% Collect your definitions here

% Back references ("Cited on page x")
\renewcommand*{\backref}[1]{} 
\renewcommand*{\backrefalt}[4]{%
\ifcase #1
\or (Cited on page~#2.)  %
\else 
(Cited on pages~#2.)  
\fi
}

% Add number-less footnotes to the start of your chapters to indicate origins
\let\svthefootnote\thefootnote
\newcommand\blankfootnote[1]{%
  \let\thefootnote\relax\footnotetext{#1}%
  \let\thefootnote\svthefootnote%
}
\let\svfootnote\footnote
\renewcommand\footnote[2][?]{%
  \if\relax#1\relax%
    \blankfootnote{#2}%
  \else%
    \if?#1\svfootnote{#2}\else\svfootnote[#1]{#2}\fi%
  \fi
}

% Some command for the writing phase
\newcommand{\todo}[1]{\textbf{\textcolor{red}{#1}}}  %todo's in red
\newcommand{\new}[1]{\textcolor{blue}{#1}} %new content in blue
\newcommand{\mdr}[1]{\textcolor{orange}{[MdR: #1]}}
\newcommand{\daan}[1]{\textcolor{violet}{\textbf{[DO: #1]}}}
\newcommand{\gab}[1]{\textcolor{magenta}{\textbf{[Gab: #1]}}}

% Define some commonly used acronyms. 
\acrodef{IR}{information retrieval}
\acrodef{ML}{Machine Learning}
\acrodef{GDPR}{General Data Protection Regulation}
\acrodef{FT}{Feature Tweaking}
\acrodef{RP}{Random Perturbation}
\acrodef{FOCUS}{Flexible Optimizable CoUnterfactual Explqanations for Tree EnsembleS}
\acrodef{FACT}{Fairness, Accountability, Confidentiality and Transparency}
\acrodef{FACT-AI}{Fairness, Accountability, Confidentiality and Transparency in Artificial Intelligence}
\acrodef{MLRC}{Machine Learning Reproducibility Challenge}
\acrodef{GNN}{Graph Neural Network}



% Define new commands
% Names
\newcommand{\OurCompany}{the retailer}
\newcommand{\OurMethod}{MC-BRP}
\newcommand{\OurUniversity}{the University of Amsterdam}
%\newcommand{\OurMethod}{\textsc{CF-GNNExplainer}}
\newcommand{\OurShort}{\textsc{CF-GNN}}
\newcommand{\synone}{\textsc{ba-shapes}}
\newcommand{\synfour}{\textsc{tree-cycles}}
\newcommand{\synfive}{\textsc{tree-grid}}
\newcommand{\gnnexplainer}{\textsc{GNNExplainer}}
\newcommand{\gnnexpshort}{\textsc{GNNExp}}
\newcommand{\pgexplainer}{\textsc{PGExplainer}}
\newcommand{\pgexpshort}{\textsc{PGExp}}
\newcommand{\mutag}{\textsc{MUTAG}}

\newcommand{\cgraph}{subgraph neighborhood}
\newcommand{\baserand}{\textsc{random}}
\newcommand{\basekeep}{\textsc{1hop}}
\newcommand{\baserm}{\textsc{rm-1hop}}


% Results tables
\newcommand{\enkelop}{$^{\vartriangle}$}
\newcommand{\dubbelop}{$^{\blacktriangle}$}
\newcommand{\enkelneer}{$^{\triangledown}$}
\newcommand{\dubbelneer}{$^{\blacktriangledown}$}
\newcommand{\notsig}{$^{\circ}$}

\newcommand{\unable}{------}
\newcommand{\NoExample}{$^\otimes$}

% Math
\newcommand\inner[2]{\langle #1, #2 \rangle}
\newcommand{\norm}[1]{\left\lVert#1\right\rVert}

\newtheorem{theorem}{Theorem}[section]
\newtheorem{lemma}[theorem]{Lemma}
\newtheorem{proposition}[theorem]{Proposition}
\newtheorem{corollary}[theorem]{Corollary}
\theoremstyle{definition}
\newtheorem{definition}{Definition}[section]
\newtheorem{example}{Example}[section]
\theoremstyle{remark}
\newtheorem{remark}{Remark}
\newcommand{\RR}{\mathbb{R}} %Reals
\newcommand{\ZZ}{\mathbb{Z}} %Integers
\newcommand{\NN}{\mathbb{N}} %Naturals
\newcommand{\CC}{\mathbb{C}} %Complex
\newcommand{\QQ}{\mathbb{Q}} %Rationals
\newcommand{\FF}{\mathbb{F}} %Field
\DeclareMathOperator*{\argmax}{argmax}
\DeclareMathOperator*{\argmin}{argmin}
\DeclareMathOperator{\softmax}{softmax}


% GCN explanation notation
\newcommand{\nodes}{\mathcal{V}}
\newcommand{\edges}{\mathcal{E}}
\newcommand{\graph}{\mathcal{G}}
\newcommand{\compgraph}{\mathcal{G}_v}
\newcommand{\compadj}{A_v}
\newcommand{\compdeg}{D_v}
\newcommand{\compnode}{X_v}
\newcommand{\Lapl}{\tilde{L}}
\newcommand{\Weights}{W^\ell}
\newcommand{\Hidden}{H^\ell}
\newcommand{\dataset}{\mathcal{D}}
\newcommand{\cfx}{\bar{x}}
\newcommand{\cfv}{\bar{v}}
\newcommand{\loss}{\mathcal{L}}
\newcommand{\losspred}{\mathcal{L}_{pred}}
\newcommand{\lossnode}{\mathcal{L}_{ndist}}
\newcommand{\lossgraph}{\mathcal{L}_{gdist}}
\newcommand{\lossdist}{\mathcal{L}_{dist}}
\newcommand{\perturbb}{P}
\newcommand{\perturbl}{\hat{P}}


% For algorithm
\newlength\myindent
\setlength\myindent{1em}
\newcommand\bindent{%
  \begingroup
  \setlength{\itemindent}{\myindent}
  \addtolength{\algorithmicindent}{\myindent}
}
\newcommand\eindent{\endgroup}

% Make Algorithm counter depend on chapter
\counterwithin{algorithm}{chapter}


% For quote
\newlength\longest

% gabbi

\newtheorem{question}{RQ}

\parskip0pt

%%% Local Variables:
%%% mode: latex
%%% TeX-master: "thesis-main"
%%% End:
